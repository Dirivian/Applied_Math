\documentclass[11pt]{article}

\usepackage{graphicx}
\usepackage{amsmath,amsfonts,amssymb}
\usepackage{enumitem}
\usepackage{hyperref}  % for urls and hyperlinks
\usepackage{listings}
\usepackage{color}

\setlength{\textwidth}{6.2in}
\setlength{\oddsidemargin}{0.3in}
\setlength{\evensidemargin}{0in}
\setlength{\textheight}{9in}
\setlength{\voffset}{-1in}
\setlength{\headsep}{26pt}
\setlength{\parindent}{0pt}
\setlength{\parskip}{5pt}

\definecolor{keywords}{RGB}{255,0,90}
\definecolor{comments}{RGB}{0,0,113}
\definecolor{red}{RGB}{160,0,0}
\definecolor{green}{RGB}{0,150,0}
\definecolor{codegreen}{rgb}{0,0.6,0}
\definecolor{codegray}{rgb}{0.5,0.5,0.5}
\definecolor{codepurple}{rgb}{0.58,0,0.82}
\definecolor{backcolour}{rgb}{0.95,0.95,0.92}
\definecolor{brown}{rgb}{0.59, 0.29, 0.0}
\definecolor{beaublue}{rgb}{0.74, 0.83, 0.9}
\definecolor{orange}{rgb}{1.0, 0.5, 0.0}
\definecolor{darkslategray}{rgb}{0.18, 0.31, 0.31}
\definecolor{deepblue}{rgb}{0,0,0.5}
\definecolor{deepred}{rgb}{0.6,0,0}
\definecolor{deepgreen}{rgb}{0,0.5,0}
\lstdefinestyle{myMatlabstyle}{
	language=Matlab,
	backgroundcolor=\color{white},   
	commentstyle=\color{codegreen},
	keywordstyle=\color{blue},
	identifierstyle=\color{brown},
	numberstyle=\tiny\color{codegray},
	stringstyle=\color{orange},
	basicstyle=\footnotesize,
	breakatwhitespace=false,         
	breaklines=true,                 
	captionpos=b,                    
	keepspaces=true,                 
	numbers=left,                    
	numbersep=5pt,                  
	showspaces=false,                
	showstringspaces=false,
	showtabs=false,                  
	tabsize=2
}
\lstdefinestyle{myPythonstyle}{
	language=Python, 
	basicstyle=\ttfamily\small, 
	keywordstyle=\color{blue},
	commentstyle=\color{green},
	stringstyle=\color{red},
	showstringspaces=false,
	identifierstyle=\color{black},
}
\lstset{language=Matlab,frame=single}
\lstset{language=Python,frame=single}

% input some useful macros from RJLmacros.tex:
\input{RJLmacros}

\begin{document}

\hfill\vbox{\hbox{AMath 574}\hbox{Homework 1}
\hbox{Jithin George, 1622555}}



%--------------------------------------------------------------------------
\vskip 1cm
\hrule
{\bf Problem \#2.7 in the book}


% uncomment the next two lines if you want to insert solution...
\vskip 1cm
{\bf Solution:}

\[ v_t - u_x=0 \]
\[ u_t - {p(v)}_x=0 \]
This is the same as
\[ u_t - p'(v)v_x=0 \]



	\[
	\begin{bmatrix}
	v\\
	u
	\end{bmatrix}_t +\begin{bmatrix}	0 & 1 \\ - p'(v) &0 \end{bmatrix} 	\begin{bmatrix}	v\\ u \end{bmatrix}_x =0 \]
	
\[\lambda^2 = - p'(v)\]
For the eigenvalues to be real, p'(v) has to be negative.
Thus, for the matrix to be diagonalizable and the system of equations to be hyperbolic,  p'(v) must be less than 0.

%--------------------------------------------------------------------------
\vskip 1cm
\hrule
{\bf Problem \#3.1 in the book} You might want to do Problem 3.2 first.


% uncomment the next two lines if you want to insert solution...
\vskip 1cm
{\bf Solution:}
\begin{enumerate}[label=(\alph*)]



	\item
	\begin{itemize}
		\item  The phase plot of q
		\begin{minipage}{\linewidth}
			\centering
			\includegraphics[width=8cm]{pha.png}
			%\caption{The phase plot }
		\end{minipage}
		
		\item  Pressure at time =1
				\begin{minipage}{\linewidth}
					\centering
					\includegraphics[width=8cm]{pra.png}
					%\caption{The phase plot }
				\end{minipage}
		\item Velocity at time =1
				\begin{minipage}{\linewidth}
					\centering
					\includegraphics[width=8cm]{va.png}
					%\caption{The phase plot }
				\end{minipage}
	\end{itemize}
		\item
	\begin{itemize}
		\item  The phase plot of q
		\begin{minipage}{\linewidth}
			\centering
			\includegraphics[width=8cm]{phb.png}
			%\caption{The phase plot }
		\end{minipage}
		
		\item  Pressure at time =1
		\begin{minipage}{\linewidth}
			\centering
			\includegraphics[width=8cm]{prb.png}
			%\caption{The phase plot }
		\end{minipage}
		\item Velocity at time =1
		\begin{minipage}{\linewidth}
			\centering
			\includegraphics[width=8cm]{vb.png}
			%\caption{The phase plot }
		\end{minipage}
	\end{itemize}
		\item
	\begin{itemize}
		\item  The phase plot of q
		\begin{minipage}{\linewidth}
			\centering
			\includegraphics[width=8cm]{phc.png}
			%\caption{The phase plot }
		\end{minipage}
		
		\item  Pressure at time =1
		\begin{minipage}{\linewidth}
			\centering
			\includegraphics[width=8cm]{prc.png}
			%\caption{The phase plot }
		\end{minipage}
		\item Velocity at time =1
		\begin{minipage}{\linewidth}
			\centering
			\includegraphics[width=8cm]{vc.png}
			%\caption{The phase plot }
		\end{minipage}
	\end{itemize}
		\item
	\begin{itemize}
		\item  The phase plot of q
		\begin{minipage}{\linewidth}
			\centering
			\includegraphics[width=8cm]{phd.png}
			%\caption{The phase plot }
		\end{minipage}
		
		\item  Pressure at time =1
		\begin{minipage}{\linewidth}
			\centering
			\includegraphics[width=8cm]{prd.png}
			%\caption{The phase plot }
		\end{minipage}
		\item Velocity at time =1
		\begin{minipage}{\linewidth}
			\centering
			\includegraphics[width=8cm]{vd.png}
			%\caption{The phase plot }
		\end{minipage}
	\end{itemize}
		\item
	\begin{itemize}
		\item  The phase plot of q
		\begin{minipage}{\linewidth}
			\centering
			\includegraphics[width=8cm]{phe.png}
			%\caption{The phase plot }
		\end{minipage}
		
		\item  Pressure at time =1
		\begin{minipage}{\linewidth}
			\centering
			\includegraphics[width=8cm]{pre.png}
			%\caption{The phase plot }
		\end{minipage}
		\item Velocity at time =1
		\begin{minipage}{\linewidth}
			\centering
			\includegraphics[width=8cm]{ve.png}
			%\caption{The phase plot }
		\end{minipage}
	\end{itemize}
		\item
		\begin{itemize}
			\item  The phase plot of q
			\begin{minipage}{\linewidth}
				\centering
				\includegraphics[width=8cm]{phf.png}
				%\caption{The phase plot }
			\end{minipage}
			
			\item  Pressure at time =1
			\begin{minipage}{\linewidth}
				\centering
				\includegraphics[width=8cm]{prf.png}
				%\caption{The phase plot }
			\end{minipage}
			\item Velocity at time =1
			\begin{minipage}{\linewidth}
				\centering
				\includegraphics[width=8cm]{vf.png}
				%\caption{The phase plot }
			\end{minipage}
		\end{itemize}			
\end{enumerate}

% insert your solution here!


%--------------------------------------------------------------------------
\vskip 1cm
\hrule
{\bf Problem \#3.2 in the book}

You can use Matlab for this one, but I suggest you try writing the program 
in Python.  A Jupyter notebook will be provided to help you get started.

Note that the module {\tt numpy.linalg} contains an {\tt eig}
function similar to Matlab.

% uncomment the next two lines if you want to insert solution...
\vskip 1cm
{\bf Solution:}

We start with the initial conditions.

\begin{lstlisting}[style=MyPythonstyle]
import numpy as np
import pylab as pl
A = np.array([[2.,2],[0,-3]])
ql = np.array([1,1])
qr = np.array([0,3])
\end{lstlisting}

Make some useful functions

\begin{lstlisting}[style=MyPythonstyle]
#Get Eigen-values/vectors
def Reisolve(A,ql,qr):
	lam,R = np.linalg.eig(A)
	T= qr-ql
	Rinv = np.linalg.inv(R)
	alpha =np.dot(Rinv,T)
	return alpha,lam ,R

#Phaseplots
def phaseplot(ql,qr,R,qm):
	k1= -3
	k2= 3
	r1 = R[:,0]
	r2 = R[:,1]
	u1 = ql+k1*r1
	u2 = ql + k2*r1
	u = np.vstack((u1,u2))
	v1 = qr+k1*r2
	v2 = qr + k2*r2
	v = np.vstack((v1,v2))
	pl.plot(ql[0],ql[1], 'ro') #ql
	pl.plot(qr[0],qr[1], 'bo') #qr
	pl.plot(qm[0],qm[1], 'ko') #qm
	pl.plot(u[:,0],u[:,1]) #First eigenvector
	pl.plot(v[:,0],v[:,1]) #Second Eigenvector
	pl.xlabel("q1")
	pl.ylabel("q2")
	
#Solve for various x at particular time
def qsolve(xcol,t,lamda,R,alpha):
	i=0
	q=np.zeros((2,len(xcol)))
	for x in xcol:
		if x/t <lamda[0]: #Naturally, there is a bug at t=0
			q[:,i]=ql
		elif x/t <lamda[1]:
			q[:,i] = ql + alpha[0]*R[:,0]
		else:
			q[:,i]=qr
		i+=1
	return q

#Solve for middle states
def qmsolve(R,alpha,ql):
	i=0
	qm=np.zeros((2,1))
	qm[:,i] = ql + alpha[0]*R[:,0]
	return qm
\end{lstlisting}

Solve and plot.
\begin{lstlisting}[style=MyPythonstyle]
al, lamda, R1= Reisolve(A,ql,qr)
ind = np.argsort(lamda)
lamda=lamda[ind]
R= R1[:,ind]
al= al[ind] 
x=np.linspace(-10,10,100)
t=1
qm=qmsolve(x,t,lamda,R,al) 
pl.figure(3)
phaseplot(ql,qr,R,qm)
pl.figure(1) 
pl.plot(x,q[0,:])
pl.xlabel("x")
pl.ylabel("q1")

pl.figure(2)
pl.plot(x,q[1,:])
pl.xlabel("x")
pl.ylabel("q2")
\end{lstlisting}
%--------------------------------------------------------------------------

\vskip 1cm
\hrule
{\bf Problem \#3.3 in the book}

Following the sort of thing done in 
script \verb+problem_3_5.py+ might be useful if you want to insert a
figure in your solution, or you can draw with another programming language, or
sketch the solution by hand and scan.


% uncomment the next two lines if you want to insert solution...
\vskip 1cm
{\bf Solution:}

The code to plot the figure is given below.
\begin{lstlisting}[style=MyPythonstyle]
from pylab import *

clf()  # clear figure

plot([-4,4],[0,0],'k') # x-axis  'k' means black line
plot([-4,-4],[0,6],'k') # t-axis
plot([4,4],[0,6],'k') # right boundary

def plot_rightgoing(x1,t1,s):
"""
plot right-going wave of slope s starting at (x1,t1) to right boundary
"""
	t2 = t1 + s*(4-x1)
	plot([x1,4], [t1,t2], 'k')

def plot_leftgoing(x1,t1,s):
"""
plot left-going wave of slope s starting at (x1,t1) to left boundary
"""
	t2 = t1 + s*(-4-x1)
	plot([x1,-4], [t1,t2], 'k')

plot_rightgoing(0,0,1)
plot_rightgoing(0,0,1)
def Reisolve(A,ql,qr):
	lam,R = linalg.eig(A)
	T= qr-ql
	Rinv = linalg.inv(R)
	alpha =dot(Rinv,T)
	return alpha,lam ,R
def qsolve(xcol,t,lamda,R,alpha):
	i=0
	q=np.zeros((2,len(xcol)))
	for x in xcol:
	if x/t <lamda[0]: #Naturally, there is a bug at t=0
		q[:,i]=ql
	elif x/t <lamda[1]:
		q[:,i] = ql + alpha[0]*R[:,0]
	else:
		q[:,i]=qr
	i+=1
	return q
A = np.array([[0,0,4],[0,1,0],[1,0,0]])
ql = np.array([1,2,0])
qr = np.array([1,5,1])
al, lamda, R1= Reisolve(A,ql,qr)
ind = np.argsort(lamda)
lamda=lamda[ind]
R= R1[:,ind] 

#Plotting the regions
for l in lamda:
	if l>0:
		plot_rightgoing(0,0,l)
	else:
		plot_leftgoing(0,0,l)
text(-3, 4, 'q_l')
text(0, 7, 'q_l*')
text(3.0, 4, 'q_r*')
text(3, 1, 'q_r')
\end{lstlisting}

\begin{enumerate}[label=(\alph*)]
	
	
	
	\item
					\begin{minipage}{\linewidth}
						\centering
						\includegraphics[width=8cm]{33a.png}
						%\caption{The phase plot }
					\end{minipage}
					
	\[q_l* = \bcm 0 \\ 2 \\ 0.5\ecm	, q_r* = \bcm 0 \\ 5 \\ 0.5\ecm	\]	
	\item
			\begin{minipage}{\linewidth}
				\centering
				\includegraphics[width=8cm]{33b.png}
				%\caption{The phase plot }
			\end{minipage}
			
	\[q_l* = \bcm 1 \\ 1 \\ 1\ecm	, q_r* = \bcm 1 \\ 3 \\ 1\ecm	\]				
\end{enumerate}

% to insert a figure named X.png, you might use this...
% \hfil\includegraphics[width=4.0in]{X.png}\hfil


%--------------------------------------------------------------------------

\vskip 1cm
\hrule
{\bf Problem \#3.5 in the book}

The script \verb+problem_3_5.py+ was used to generate this figure:

% you need to run
%     python problem_3_5.py
% at the command line to generate the figure inserted here:
\hfil\includegraphics[width=3.5in]{problem_3_5.png}\hfil

To solve this problem, determine the states $A,~ B, ~ \ldots,~ M$ and also
the times $t_1,~t_2,~t_3$.  The times can be written in terms of the
parameters $\rho_0$ and $K_0$, which were not stated in the problem.

For example,
\[
A = \bcm 0 \\ 0 \ecm, \quad, B = \bcm 1 \\ 0 \ecm, \quad, 
C = \bcm 0 \\ 0 \ecm, \quad \ldots
\]

% Note that bcm and ecm (begin and end centered matrices) are defined in 
% RJLmacros.tex

% uncomment the next two lines if you want to insert solution...
\vskip 1cm


{\bf Solution:}

From the initial conditions, we have 
\[
A = \bcm 0 \\ 0 \ecm, \quad, B = \bcm 1 \\ 0 \ecm, \quad, 
C = \bcm 0 \\ 0 \ecm
\]
The eigenvalues are $\sqrt{K_0/\rho_0} $ and - $\sqrt{K_0/\rho_0} $.
\[t_1 = \frac{\Delta x}{-\lambda } = \sqrt{\frac{\rho_0}{K_0}}\]
\[t_2 = 2\sqrt{\frac{\rho_0}{K_0}}\]
\[t_3 = 3\sqrt{\frac{\rho_0}{K_0}}\]
The eigenvector matrix and it's inverse are
\[
R = \bcm -Z_0 & Z_0\\ 1 & 1 \ecm, \quad, R^{-1} = \frac{1}{2Z_0}\bcm -1 &Z_0 \\ 1 & Z_0 \ecm \quad, 
\]
where 
\[Z_0 = \sqrt{K_0\rho_0}\]
D:
\[\bcm a_1\\ a_2 \ecm = R^{-1}(B-A)= \frac{1}{2Z_0}\bcm -1 &Z_0 \\ 1 & Z_0 \ecm \bcm 1 \\ 0 \ecm= \frac{1}{2Z_0}\bcm -1 \\ 1  \ecm \]

\[W_1 = a_1 R_1= \frac{1}{2Z_0}\bcm Z_0 \\ -1  \ecm  \]
\[W_2 = a_2 R_2= \frac{1}{2Z_0}\bcm Z_0 \\ 1  \ecm  \]

\[ D = \bcm 0 \\ 0 \ecm + a_1 R_1=\bcm \frac{1}{2} \\ \frac{-1}{2Z_0}  \ecm  \]
E:
\[\bcm a_3\\ a_4 \ecm = R^{-1}(C-B)= \frac{1}{2Z_0}\bcm -1 &Z_0 \\ 1 & Z_0 \ecm \bcm- 1 \\ 0 \ecm= \frac{1}{2Z_0}\bcm 1 \\ -1  \ecm \]

\[ E = \bcm 1 \\ 0 \ecm + a_1 R_1=\bcm \frac{1}{2} \\ \frac{1}{2Z_0}  \ecm  \]
\[W_3 = a_3 R_1= \frac{1}{2Z_0}\bcm -Z_0 \\ 1  \ecm  \]
\[W_4 = a_4 R_2= \frac{1}{2Z_0}\bcm -Z_0 \\ -1  \ecm  \]
F:
At the boundary, the left going wave $W_1$ reflects and we have the new wave $W'_1$
\[ W'_1 = -W_1= \frac{1}{2Z_0}\bcm -Z_0 \\ 1  \ecm \]
\[F= D-W'_1= D+W_1= \bcm 1 \\ \frac{-1}{Z_0}  \ecm \]
G:
\[G= D+W'_1= D-W_1= \bcm 0 \\ 0  \ecm \]
H:
\[H= E+W_1= \bcm 1 \\ 0  \ecm \]

I:
\[I= F+W_3=\frac{1}{2Z_0} \bcm Z_0 \\ -1  \ecm \]
J:
\[J= G+W'_4=G-W_4=  \frac{1}{2Z_0}\bcm Z_0 \\ 1  \ecm \]
K:
\[K= I-W'_3=I+W_3=  \bcm 0 \\0  \ecm \]
L:
\[L= I+W'_4=I-W_4=  \bcm 1 \\ \frac{-1}{Z_0} \ecm \]
M:
\[M= J+W'_2=J-W_2=  \bcm 0 \\0  \ecm \]


\hfil\includegraphics[width=6in]{1}\hfil

\vskip 1cm
\hrule
{\bf Problem \#3.5A}
Solve \#3.5 with {\em periodic} boundary conditions instead of reflecting
walls.  Sketch the solution in the $x$--$t$ plane
up to at least time $t_3$ (as in \#3.5 the time
the right-going wave from $x_0=1$ hits the right boundary) 
and indicate the state in each section.  You might want to modify the
script \verb+problem_3_5.py+ to make the plot.


% uncomment the next two lines if you want to insert solution...
\vskip 1cm
{\bf Solution:}

With periodic boundary conditions, we get
\[ W_2(0,t)= W_1(4,t)\]
\[ W_1(4,t)= W_2(0,t)\]
And this following image is obtained.

\hfil\includegraphics[width=3.5in]{BOUND.png}\hfil

A,B,C,D and E remain the same.

\[
A = \bcm 0 \\ 0 \ecm, \quad, B = \bcm 1 \\ 0 \ecm, \quad, 
C = \bcm 0 \\ 0 \ecm
\]
The eigenvalues are $\sqrt{K_0/\rho_0} $ and - $\sqrt{K_0/\rho_0} $.
\[t_1 = \frac{\Delta x}{-\lambda } = \sqrt{\frac{\rho_0}{K_0}}\]
\[t_2 = 2\sqrt{\frac{\rho_0}{K_0}}\]
\[t_3 = 3\sqrt{\frac{\rho_0}{K_0}}\]
The eigenvector matrix and it's inverse are
\[
R = \bcm -Z_0 & Z_0\\ 1 & 1 \ecm, \quad, R^{-1} = \frac{1}{2Z_0}\bcm -1 &Z_0 \\ 1 & Z_0 \ecm \quad, 
\]
where 
\[Z_0 = \sqrt{K_0\rho_0}\]

D:
\[\bcm a_1\\ a_2 \ecm = R^{-1}(B-A)= \frac{1}{2Z_0}\bcm -1 &Z_0 \\ 1 & Z_0 \ecm \bcm 1 \\ 0 \ecm= \frac{1}{2Z_0}\bcm -1 \\ 1  \ecm \]

\[W_1 = a_1 R_1= \frac{1}{2Z_0}\bcm Z_0 \\ -1  \ecm  \]
\[W_2 = a_2 R_2= \frac{1}{2Z_0}\bcm Z_0 \\ 1  \ecm  \]

\[ D = \bcm 0 \\ 0 \ecm + a_1 R_1=\bcm \frac{1}{2} \\ \frac{-1}{2Z_0}  \ecm  \]
E:
\[\bcm a_3\\ a_4 \ecm = R^{-1}(C-B)= \frac{1}{2Z_0}\bcm -1 &Z_0 \\ 1 & Z_0 \ecm \bcm- 1 \\ 0 \ecm= \frac{1}{2Z_0}\bcm 1 \\ -1  \ecm \]

\[ E = \bcm 1 \\ 0 \ecm + a_1 R_1=\bcm \frac{1}{2} \\ \frac{1}{2Z_0}  \ecm  \]
\[W_3 = a_3 R_1= \frac{1}{2Z_0}\bcm -Z_0 \\ 1  \ecm  \]
\[W_4 = a_4 R_2= \frac{1}{2Z_0}\bcm -Z_0 \\ -1  \ecm  \]

F:

\[F= C+ W_1= \frac{1}{2Z_0}\bcm Z_0 \\ -1  \ecm \]

G:
\[G= D+ W_3= \bcm 0 \\ 0  \ecm \]

H:
\[H= E+ W_1= \bcm 1 \\ 0  \ecm \]

%--------------------------------------------------------------------------
\vskip 1cm
\hrule
{\bf Problem \#4.1 in the book}


% uncomment the next two lines if you want to insert solution...
\vskip 1cm
{\bf Solution:}



	\[\lambda_1 = u_0 -c_0\]
		\[\lambda_2 = u_0 +c_0\]
\begin{itemize}
	\item
If the flow is subsonic, there is one positive eigenvalue and one negative one.
\[A^+ = R\lambda_2R^{-1} = \bcm -Z_0 & Z_0\\ 1 & 1 \ecm \bcm 0 &0\\0 &(u_0 +c_0)\ecm  \frac{1}{2Z_0}\bcm -1 &Z_0 \\ 1 & Z_0 \ecm =\frac{1}{2Z_0}\bcm Z_0\lambda_2 &Z_0^2 \lambda_2  \\ \lambda_2 & Z_0 \lambda_2 \ecm\]
\[A^- = R\lambda_1R^{-1} = \bcm -Z_0 & Z_0\\ 1 & 1 \ecm \bcm (u_0-c_0) &0\\0 &0\ecm  \frac{1}{2Z_0}\bcm -1 &Z_0 \\ 1 & Z_0 \ecm =\frac{1}{2Z_0}\bcm Z_0\lambda_1 &Z_0^2 \lambda_1  \\ \lambda_1 & Z_0 \lambda_1 \ecm\]
 If $Q_i$ and $Q_{i-1}$ are given, let
 \[\Delta Q_{i-1/2}= Q_i - Q_{i-1}\]
 \[W^1_{i-1/2}=\frac{1}{\lambda_1} A^-\Delta Q_{i-1/2}\]
 \[W^2_{i-1/2}=\frac{1}{\lambda_2} A^+\Delta Q_{i-1/2}\]
\item
If $u_0>c_0$, both eigenvalues are positive,
\[A^+=R\lambda R^{-1}= A\]
\[A^- = \bcm 0 &0\\ 0 &0 \ecm \]
\item 
If $u_0<-c_0$, 
\[A^+ =\bcm 0 &0\\ 0 &0 \ecm \]
\[A^- =R\lambda R^{-1}= A\]

For the last two cases,
 \[ \bcm a_1 \\ a_2 \ecm = R^{-1}\Delta Q_{i-1/2}\]
  \[W^1_{i-1/2}= a_1 R_1\]
  \[W^2_{i-1/2}=a_2 R_2\]

	\end{itemize}


%--------------------------------------------------------------------------
\vskip 1cm
\hrule
{\bf Problem \#4.2 in the book}


% uncomment the next two lines if you want to insert solution...
\vskip 1cm
{\bf Solution:}

\begin{enumerate}[label=(\alph*)]
	\item
	
	
	
\hfil\includegraphics[width=3in]{2}\hfil

	
\hfil\includegraphics[width=6.5in]{3}\hfil


\hfil\includegraphics[width=6.5in]{4}\hfil
\end{enumerate}


% insert your solution here!

\end{document}

