\documentclass[a4paper,11pt]{article}

\usepackage{fullpage} % Package to use full page
\usepackage{parskip} % Package to tweak paragraph skipping
\usepackage{amsmath}
\usepackage{hyperref}
\usepackage{amsmath,amsfonts,amsthm} % Math packages
\usepackage{graphicx}
\usepackage{listings}
\usepackage{caption}
\usepackage{subcaption}
\usepackage{color}
\usepackage{float}
\definecolor{codegreen}{rgb}{0,0.6,0}
\definecolor{codegray}{rgb}{0.5,0.5,0.5}
\definecolor{codepurple}{rgb}{0.58,0,0.82}
\definecolor{backcolour}{rgb}{0.95,0.95,0.92}
\definecolor{brown}{rgb}{0.59, 0.29, 0.0}
\definecolor{beaublue}{rgb}{0.74, 0.83, 0.9}
\definecolor{orange}{rgb}{1.0, 0.5, 0.0}
\definecolor{darkslategray}{rgb}{0.18, 0.31, 0.31}
\def\Xint#1{\mathchoice
	{\XXint\displaystyle\textstyle{#1}}%
	{\XXint\textstyle\scriptstyle{#1}}%
	{\XXint\scriptstyle\scriptscriptstyle{#1}}%
	{\XXint\scriptscriptstyle\scriptscriptstyle{#1}}%
	\!\int}
\def\XXint#1#2#3{{\setbox0=\hbox{$#1{#2#3}{\int}$}
		\vcenter{\hbox{$#2#3$}}\kern-.5\wd0}}
\def\dashint{\Xint-}

% Swap the definition of \abs* and \norm*, so that \abs
% and \norm resizes the size of the brackets, and the 
% starred version does not.
\makeatletter
\let\oldabs\abs
\def\abs{\@ifstar{\oldabs}{\oldabs*}}
%
\let\oldnorm\norm
\def\norm{\@ifstar{\oldnorm}{\oldnorm*}}
\makeatother
\definecolor{keywords}{RGB}{255,0,90}
\definecolor{comments}{RGB}{0,0,113}
\definecolor{red}{RGB}{160,0,0}
\definecolor{green}{RGB}{0,150,0}
\definecolor{codegreen}{rgb}{0,0.6,0}
\definecolor{codegray}{rgb}{0.5,0.5,0.5}
\definecolor{codepurple}{rgb}{0.58,0,0.82}
\definecolor{backcolour}{rgb}{0.95,0.95,0.92}
\definecolor{brown}{rgb}{0.59, 0.29, 0.0}
\definecolor{beaublue}{rgb}{0.74, 0.83, 0.9}
\definecolor{orange}{rgb}{1.0, 0.5, 0.0}
\definecolor{darkslategray}{rgb}{0.18, 0.31, 0.31}
\definecolor{deepblue}{rgb}{0,0,0.5}
\definecolor{deepred}{rgb}{0.6,0,0}
\definecolor{deepgreen}{rgb}{0,0.5,0}
\lstdefinestyle{myMatlabstyle}{
	language=Matlab,
	backgroundcolor=\color{white},   
	commentstyle=\color{codegreen},
	keywordstyle=\color{blue},
	identifierstyle=\color{brown},
	numberstyle=\tiny\color{codegray},
	stringstyle=\color{orange},
	basicstyle=\footnotesize,
	breakatwhitespace=false,         
	breaklines=true,                 
	captionpos=b,                    
	keepspaces=true,                 
	numbers=left,                    
	numbersep=5pt,                  
	showspaces=false,                
	showstringspaces=false,
	showtabs=false,                  
	tabsize=2
}
\lstdefinestyle{myPythonstyle}{
	language=Python, 
	basicstyle=\ttfamily\small, 
	keywordstyle=\color{blue},
	commentstyle=\color{green},
	stringstyle=\color{red},
	showstringspaces=false,
	identifierstyle=\color{black},
}
\lstset{language=Matlab,frame=single}
\lstset{language=Python,frame=single}

\title{AMATH 562: Homework 1}
\author{Jithin D. George, No. 1622555}
%\date{23/11/16}
% matrix environment
\newenvironment{mat}{\left[ \begin{array}{ccccccccccccc}}{\end{array}\right]}
\newcommand\bcm{\begin{mat}}
	\newcommand\ecm{\end{mat}}

\begin{document}

\maketitle
\begin{enumerate}

\item {\bf 6.2}

Let us define
\[A := \{s\in S : \lim_{n \to \infty} |X_n(s)-X(s)|=0\}\]
\[A := \{s\in S : \lim_{n \to \infty} |X_n(s)-s|=0\}\]
For all s $\in (0,1)$,
\[\lim_{n \to \infty} X_n(s) = s\]
Even 
\[\lim_{n \to \infty} X_n(1) = 1\]
But,
\[\lim_{n \to \infty} X_n(0) = 0\]
Thus,
\[A = S-\{0\}\]
Since the probability distribution is uniform and so, the probability at any particular point is zero,
\[P(A)=1\]
Thus, the sequence $X_n$ converges almost surely to X under this probability measure.

\item
{\bf 6.3} 

We know there exists A and B such that
\[A := \{s\in S : \lim_{n \to \infty} |X_n(s)-X(s)|=0\}\]
\[B := \{s\in S : \lim_{n \to \infty} |Y_n(s)-Y(s)|=0\}\]
and P(A)=P(B)=1
\[\lim_{n \to \infty} |X_n(s)-X(s)+ Y_n(s)-Y(s)|\leq\lim_{n \to \infty} |X_n(s)-X(s)|+|Y_n(s)-Y(s)| \]
So,
\[C := \{s\in S : \lim_{n \to \infty} |X_n(s)+Y_n(s)-X(s)-Y(s)|=0\} \supseteq A \cap B\]
\[P(C) \geq P(A\cap B) = P(A)+P(B)-P(A\cup B)=1\]
Thus, the sequence $X_n + Y_n$ converges almost surely to X+Y under this probability measure.
\item
{\bf 6.6}
\begin{enumerate}
\item
For convergence in probability to 0, we need
\[\lim_{n \to \infty} P(|X_n|>\epsilon)=0\]
\[\lim_{n \to \infty} P(|X_n|>\epsilon)=\lim_{n \to \infty} P(|X_n|=1)=\lim_{n \to \infty} p_n\]

Thus by definition, for this problem, convergence in probability occurs if and only if $p_n$ go to zero.
\item
Since this converges in probability, we can get almost sure convergence by Theorem 6.6.2, if 
\[\sum_{n}^{\infty} P(|X_n|>\epsilon) <\infty\]

\[\sum_{n}^{\infty}  P(|X_n|>\epsilon)= \sum_{n}^{\infty}  P(|X_n|=1)=\sum_{n}^{\infty}  p_n  <\infty\]

Now, we have to show if $X_n$ converges to 0 almost surely, then $\sum_{n}^{\infty}  p_n  <\infty$.

We try to prove the contra-positive.
If $\sum_{n}^{\infty}  p_n  = \infty$, then $X_n$ does not converge to 0.
Since the events are independent, by Borel Cantelli,
\[P(|X_n|>\epsilon,i.o)=1\]
So,
\[P(\limsup_{n \to \infty}|X_n|>\epsilon)=0\]
which is the same as 
\[P(\liminf_{n \to \infty}|X_n|<\epsilon)=1\]

which is the definition of almost sure convergence (From Wikipedia).
\end{enumerate}
\item {\bf 6.7}

For $\limsup_{n \to \infty} \frac{log X_n}{log n}= C$, we need
\[P\bigg(\frac{log X_n}{log n} <C +\epsilon, a.b.f.m \bigg)=1\]
\[P\bigg(\frac{log X_n}{log n} <C -\epsilon, i.o \bigg)=1\]
Let $Y_n = \frac{log X_n}{log n}$
\[P(Y_n> y)= P(\frac{log X_n}{log n}>y)= P(X_n>n^y)=\frac{1}{n^{5y}}\]
If y$>\frac{1}{5}$
\[\sum_{n}^{\infty}P(Y_n> y)<\infty\]
Thus, by Borel-Cantelli, for y$>\frac{1}{5}$,
\[P(Y_n> y, i.o)=0\]
So,
\[P(Y_n \leq y, a.b.f.m)=1\]
\begin{equation}
P(Y_n < \frac{1}{5} + \epsilon, a.b.f.m)=1
\end{equation}
If y$\leq\frac{1}{5}$
\[\sum_{n}^{\infty}P(Y_n\leq y)=\infty\]
Thus, by Borel-Cantelli, for y$\leq\frac{1}{5}$,
\[P(Y_n> y, i.o)=1\]

\begin{equation}
P(Y_n > \frac{1}{5} - \epsilon, i.o)=1
\end{equation}

From [1] and [2], we can say
\[\limsup_{n \to \infty} \frac{log X_n}{log n}= C=\frac{1}{5}\]


\end{enumerate}  
\end{document}