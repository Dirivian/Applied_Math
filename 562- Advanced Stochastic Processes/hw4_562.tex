\documentclass[a4paper,11pt]{article}

\usepackage{fullpage} % Package to use full page
\usepackage{parskip} % Package to tweak paragraph skipping
\usepackage{amsmath}
\usepackage{hyperref}
\usepackage{amsmath,amsfonts,amsthm} % Math packages
\usepackage{graphicx}
\usepackage{listings}
\usepackage{caption}
\usepackage{subcaption}
\usepackage{color}
\usepackage{float}
\definecolor{codegreen}{rgb}{0,0.6,0}
\definecolor{codegray}{rgb}{0.5,0.5,0.5}
\definecolor{codepurple}{rgb}{0.58,0,0.82}
\definecolor{backcolour}{rgb}{0.95,0.95,0.92}
\definecolor{brown}{rgb}{0.59, 0.29, 0.0}
\definecolor{beaublue}{rgb}{0.74, 0.83, 0.9}
\definecolor{orange}{rgb}{1.0, 0.5, 0.0}
\definecolor{darkslategray}{rgb}{0.18, 0.31, 0.31}
\def\Xint#1{\mathchoice
	{\XXint\displaystyle\textstyle{#1}}%
	{\XXint\textstyle\scriptstyle{#1}}%
	{\XXint\scriptstyle\scriptscriptstyle{#1}}%
	{\XXint\scriptscriptstyle\scriptscriptstyle{#1}}%
	\!\int}
\def\XXint#1#2#3{{\setbox0=\hbox{$#1{#2#3}{\int}$}
		\vcenter{\hbox{$#2#3$}}\kern-.5\wd0}}
\def\dashint{\Xint-}

% Swap the definition of \abs* and \norm*, so that \abs
% and \norm resizes the size of the brackets, and the 
% starred version does not.
\makeatletter
\let\oldabs\abs
\def\abs{\@ifstar{\oldabs}{\oldabs*}}
%
\let\oldnorm\norm
\def\norm{\@ifstar{\oldnorm}{\oldnorm*}}
\makeatother
\definecolor{keywords}{RGB}{255,0,90}
\definecolor{comments}{RGB}{0,0,113}
\definecolor{red}{RGB}{160,0,0}
\definecolor{green}{RGB}{0,150,0}
\definecolor{codegreen}{rgb}{0,0.6,0}
\definecolor{codegray}{rgb}{0.5,0.5,0.5}
\definecolor{codepurple}{rgb}{0.58,0,0.82}
\definecolor{backcolour}{rgb}{0.95,0.95,0.92}
\definecolor{brown}{rgb}{0.59, 0.29, 0.0}
\definecolor{beaublue}{rgb}{0.74, 0.83, 0.9}
\definecolor{orange}{rgb}{1.0, 0.5, 0.0}
\definecolor{darkslategray}{rgb}{0.18, 0.31, 0.31}
\definecolor{deepblue}{rgb}{0,0,0.5}
\definecolor{deepred}{rgb}{0.6,0,0}
\definecolor{deepgreen}{rgb}{0,0.5,0}
\lstdefinestyle{myMatlabstyle}{
	language=Matlab,
	backgroundcolor=\color{white},   
	commentstyle=\color{codegreen},
	keywordstyle=\color{blue},
	identifierstyle=\color{brown},
	numberstyle=\tiny\color{codegray},
	stringstyle=\color{orange},
	basicstyle=\footnotesize,
	breakatwhitespace=false,         
	breaklines=true,                 
	captionpos=b,                    
	keepspaces=true,                 
	numbers=left,                    
	numbersep=5pt,                  
	showspaces=false,                
	showstringspaces=false,
	showtabs=false,                  
	tabsize=2
}
\lstdefinestyle{myPythonstyle}{
	language=Python, 
	basicstyle=\ttfamily\small, 
	keywordstyle=\color{blue},
	commentstyle=\color{green},
	stringstyle=\color{red},
	showstringspaces=false,
	identifierstyle=\color{black},
}
\lstset{language=Matlab,frame=single}
\lstset{language=Python,frame=single}

\title{AMATH 562: Homework 4}
\author{Jithin D. George, No. 1622555}
%\date{23/11/16}
% matrix environment
\newenvironment{mat}{\left[ \begin{array}{ccccccccccccc}}{\end{array}\right]}
\newcommand\bcm{\begin{mat}}
	\newcommand\ecm{\end{mat}}

\begin{document}

\maketitle
\begin{enumerate}

\item 
\[dX_t = \kappa(\theta -X_t)dt + \delta \sqrt{X_t} dW_t\]
\[u(t,x) = \mathbb{E}\bigg[e^{- \int_t^T X_s ds}|X_t = x\bigg]\]
By Theorem 9.2.2, we can obtain the PDE
\[(\partial_t + \mathcal{A}(t))u +g = 0\]
where 
\[g = 0\]
 and 
\[\mathcal{A}(t)  =\kappa(\theta -x)\partial_x+\frac{\delta^2 x}{2}\partial_x^2 - x\]
then the boundary is given by
\[u(T,x) = 1\]
We use the following ansatz
\[u(t,x)= e^{-xA(t)-B(t)}\]

Plugging it into the PDE, we get
\[(-xA'-B')u-\kappa(\theta -x)Au+\frac{\delta^2 xA^2}{2}u-xu=0\]
Thus,
\[-xA'-B'-\kappa(\theta -x)A+\frac{\delta^2 x A^2}{2}-x=0\]
\[x(-A'+\kappa\theta A+\frac{\delta^2 A^2}{2}-1)+(-B'-\kappa\theta A)=0\]
Thus, we can get the following coupled odes.
\[A'=\kappa\theta A+\frac{\delta^2 A^2}{2}-1\]
\[B'=-\kappa\theta A\]
Boundary conditions:
\[u(T,x)=1\]
\[-xA(T)-B(T)=0\]
Since this should hold for all x,
\[A(T)=B(T)=0\]

\[A(t) = \frac{\sqrt{-2\delta^2-\kappa^2\theta^2}\tan(\frac{1}{2}(c_1\sqrt{-2\delta^2-\kappa^2\theta^2}+t\sqrt{-2\delta^2-\kappa^2\theta^2} ) )-\kappa\theta}{\delta^2}\]
where
\[c_1 = \frac{2}{\sqrt{-2\delta^2-\kappa^2\theta^2}}\tan^{-1}\bigg(\frac{\kappa\theta}{\sqrt{-2\delta^2-\kappa^2\theta^2}}\bigg)-T\]

\[B(t) = \frac{2\log sech \frac{|c_1\sqrt{-2\delta^2-\kappa^2\theta^2}+t\sqrt{-2\delta^2-\kappa^2\theta^2} )|}{ |c_1\sqrt{-2\delta^2-\kappa^2\theta^2}+T\sqrt{-2\delta^2-\kappa^2\theta^2} )|}-\kappa\theta (t-T)}{\delta^2}\]

\item 
\[dX_t^i = -\frac{b}{2}X_t^{i}dt+ \frac{1}{2} \sigma dW_t^i\]
\[B_t = \sum_i^d \int_0^t \frac{1}{\sqrt{R_s}}X_s^{i}dW_s^{i}\]
\[B_0 = \sum_i^d \int_0^0 \frac{1}{\sqrt{R_s}}X_s^{i}dW_s^{i}=0\]
Since $B_t$ has only $dW_s$ terms , it is a martingale.
Also, since $B_t$ is an Ito process, it has continuous sample paths.
\[dB_t = \sum_i^d \frac{1}{\sqrt{R_s}}X_s^{i}dW_s^{i}\] 
\[dB_t^2 = \sum_i^d \frac{1}{R_s}X_s^{2^{i}}dt\]
\[dB_t^2 = \frac{1}{R_s} \sum_i^d X_s^{2^{i}}dt\]
\[dB_t^2 = \frac{1}{R_s} R_s dt\]
\[dB_t^2 = dt\]
Thus, since the quadratic variation is t, using Levy's characterization theorem, $B_t$ is a Brownian motion.

\[R_t = \sum_{i=1}^d(X_t^{(i)})^2\]
\[dR_t = \sum_i 2X_t^{i}dX_t^{i} + \frac{\sigma^2}{4}dt\]
\[ = \sum_i 2X_t^{i}(-\frac{b}{2}X_t^{i}dt+ \frac{1}{2} \sigma dW_t^i) + \frac{\sigma^2}{4}dt\]
\[ = \sum_i (-bX_t^{2^i} + \frac{\sigma^2}{4}) dt + X_t^{i}\sigma dW_t^i\]
\[ = \sqrt{R_s}\sigma dB_t +\sum_i (-bX_t^{2^i} + \frac{\sigma^2}{4}) dt \]


\item 

\[dX_t  = \mu X_tdt + \sigma X_t dW_t\]

\[Z_t = \log(X_t)\]
\[dZ_t = \frac{1}{X_t}dX_t-\frac{1}{2X_t^2}\sigma^2X_t^2dt\]
\[dZ_t = \mu dt + \sigma  dW_t-\frac{1}{2}\sigma^2dt\]
\[dZ_t = \underbrace{(\mu -\frac{1}{2}\sigma^2)}_a dt + \sigma  dW_t\]
The transition probability for Z , $\Gamma_z$ satisfies the following pde.
\[(\partial_t + \mathcal{A}_z)\Gamma_z =0 \]
\[\mathcal{A}_z= (\mu -\frac{1}{2}\sigma^2) \partial_z + \frac{\sigma^2}{2}\partial_z^2\]


Usually, we would expect eigenfunctions of the form $e^{wz}$ but that wouldn't work at the boundaries where we need 
\[\psi_n(\log(l))=\psi_n(\log(r))=0\]
So, we say 
\[\psi_n(z)=Ce^{wz} \sin\bigg(\frac{n\pi (z - \log(l))}{\log(r)-\log(l)}\bigg)\]
where C is the normalization constant
\[\mathcal{A}_z\psi_n(z)= aw\psi_n(z)+ \frac{a n\pi}{\log(r/l)} e^{wz} cos \bigg(\frac{n\pi (z - \log(l))}{\log(r/l)}\bigg)\]
\[+ \frac{1}{2}\sigma^2\bigg( w^2\psi_n(z)+ \frac{2n\pi w}{\log(r/l)} e^{wz} cos \bigg(\frac{n\pi (z - \log(l))}{\log(r/l)}\bigg)- \bigg(\frac{ n\pi}{\log(r/l)}\bigg)^2\psi_n(z)\bigg)\]
For $\mathcal{A}_z\psi_n(z)= \lambda \psi_n(z)$,
\[a+\sigma^2w=0\]
\[w= \frac{-a}{\sigma^2}= \frac{\frac{1}{2}\sigma^2-\mu}{\sigma^2}\]
and 
\[\lambda_n= aw+\frac{1}{2}\sigma^2(w^2-\bigg(\frac{ n\pi}{\log(r/l)}\bigg)^2)\]
Since $aw  = - \sigma^2 w^2$,
\[\lambda_n=- \frac{1}{2}\sigma^2(w^2+\bigg(\frac{ n\pi}{\log(r/l)}\bigg)^2)\]

\[m(y)= \frac{2}{\sigma^2} e^{\int dy \frac{2\mu -\sigma^2}{\sigma^2}}= \frac{2}{\sigma^2} e^{y \frac{2\mu -\sigma^2}{\sigma^2}} = \frac{2}{\sigma^2}e^{-2wy}\]

\[\langle \psi_n,\psi_n\rangle_m =1\]
\[\int_{\log{l}}^{\log{r}} \psi_n(z) \psi_n(z) m(z)= 1\]
\[C^2 \int_{\log{l}}^{\log{r}} e^{wz} \sin\bigg(\frac{n\pi (z - \log(l))}{\log(r)-\log(l)}\bigg) e^{wz} \sin\bigg(\frac{n\pi (z - \log(l))}{\log(r)-\log(l)}\bigg)\frac{2}{\sigma^2}e^{-2wz} \]
\[=\frac{2C^2}{\sigma^2} \int_{\log{l}}^{\log{r}} \sin\bigg(\frac{n\pi (z - \log(l))}{\log(r)-\log(l)}\bigg)   \sin\bigg(\frac{n\pi (z - \log(l))}{\log(r)-\log(l)}\bigg) = \frac{C^2}{\sigma^2}\log(r/l) =1\]
\[C = \frac{\sigma}{\sqrt{\log(r/l)}}\]

Thus, the normalized eigenfunctions are
\[\psi_n(z)=\frac{\sigma}{\sqrt{\log(r/l)}} e^{wz} \sin\bigg(\frac{n\pi z - \log(l)}{\log(r)-\log(l)}\bigg)\]


\[\Gamma_z(t,x,T,y)=m(y) \sum_n e^{(T-t) \lambda_n}\psi_n(y)\psi_n(x)\]
\[=\frac{2}{\sigma^2}e^{-2wy} \sum_n e^{(T-t) \lambda_n}\psi_n(y)\psi_n(x)\]
\[\Gamma_z(t,x,T,y)=\frac{2}{\sigma^2}e^{-2wy} \sum_n e^{(T-t) \lambda_n}\psi_n(\log(y))\psi_n(\log(x))\]
\[\Gamma_z(T,x,T,y)=\delta_y\]
\[P(X_T \leq y| X_t=x) = \int_l^y \Gamma_x(t,x,T,s)ds\]
\[P(X_T \leq y| X_t=x) =P(Z_T \leq \log y| Z_t=\log x) = \int_{\log l}^{\log y} \Gamma_z(t,\log x,T,s)ds\]

\[\int_l^y \Gamma_x(t,x,T,s)ds =\int_{\log l}^{\log y} \Gamma_z(t,\log x,T,s)ds\]

By Fundamental Theorem of Calculus,
\[\Gamma_x(t,x,T,y) =\frac{1}{y} \Gamma_z(t,\log x,T,\log y)=\frac{2}{\sigma^2y}e^{-2wy} \sum_n e^{(T-t) \lambda_n}\psi_n(\log y)\psi_n(\log x)\]


\item 
\begin{enumerate}
\item 
We can see that 
\[\lambda = g = 0\]
Thus, u satisfies
\[(\partial_t + \mathcal{A})u =0 \]
where
\[\mathcal{A} = \frac{1}{2}\partial_x^2+ \frac{1}{2}\partial_y^2\]
and on the boundary
\[u(x,y)= u(X_\tau,a)\]

\item
\[\phi(x) = \frac{1}{2\pi}\int e^{iwx}\hat{\phi}(w)dw\]
\[u(x,y) =\mathbb{E}[\phi(X_\tau)|X_t=x, Y_t=y ]= \mathbb{E}[\frac{1}{2\pi}\int e^{iwX_\tau}\hat{\phi}(w)dw|X_t=x, Y_t=y ] \]
\[= \mathbb{E}[\frac{1}{2\pi}\int e^{iwX_\tau}\hat{\phi}(w)dw|X_t=x, Y_t=y ] \]
\[= \frac{1}{2\pi} \int \hat{\phi}(w)  \mathbb{E}[ e^{iwX_\tau}dw|X_t=x, Y_t=y ] \]
\[= \frac{1}{2\pi}\int \hat{\phi}(w) \mathbb{E}[\mathbb{E}[ e^{iwX_\tau}dw|X_t=x, Y_t=y, \tau ]|X_t=x, Y_t=y ] \]
\[= \frac{1}{2\pi}\int \hat{\phi}(w) \mathbb{E}[\mathbb{E}[ e^{iwX_\tau}dw|X_t=x, Y_t=y, \tau ]|X_t=x, Y_t=y ] \]

$X_\tau$  is normally distributed with mean x (starting value) and variance $(\tau-t)$.
\[= \frac{1}{2\pi}\int \hat{\phi}(w)  \mathbb{E}[e^{iwx -\frac{1}{2}w^2(\tau-t)}|X_t=x, Y_t=y ] \]
\[= \frac{1}{2\pi}\int \hat{\phi}(w)  e^{iwx }\mathbb{E}[e^{-\frac{1}{2}w^2(\tau-t)}|X_t=x, Y_t=y ] \]

For hitting time $\tau_m$, we know
\[\mathbb{E}[e^{-\lambda\tau_a}]= e^{-|a| \sqrt{2\lambda}}\]
since we start from y at time t,
\[\mathbb{E}[e^{-\lambda(\tau_a-t)}]= e^{-|a-y| \sqrt{2\lambda}}\]
Thus,
\[u(x)= \frac{1}{2\pi}\int \hat{\phi}(w)  e^{iwx }e^{-|a-y||w|}dw \]
For u to be the transition probability P, $\phi(x)$ should be the indicator function $\mathbb{I}_{dz}$.
\[\hat{\phi}(w) = \int_{-\infty}^{\infty} \mathbb{I}_{dz} e^{-iwx} dx \]
\[= \int_{z}^{z+dz} \mathbb{I}_{dz} e^{-iwx} dx \]
\[= e^{-iwz} dz \]
\[P= \frac{1}{2\pi}\int e^{-iwz} dz  e^{iwx }e^{-|a-y||w|}dw \]
\[= \frac{1}{2\pi}\int e^{-iwz} dz  e^{iwx }e^{-|a-y||w|}dw \]
\[P= \frac{1}{2\pi}\int e^{-iwz} dz  e^{iwx }e^{-|a-y||w|}dw \]
\[= \frac{1}{2\pi}\int_{-\infty}^0 e^{-iwz} dz  e^{iwx }e^{|a-y|w}dw + \frac{1}{2\pi}\int_0^{\infty} e^{-iwz} dz  e^{iwx }e^{-|a-y|w}dw \]
\[=\frac{dz}{2\pi}\bigg(\frac{1}{|a-y|+i(x-z)}+\frac{1}{|a-y|-i(x-z)}\bigg)\]
\[=\frac{dz}{\pi}\bigg(\frac{|a-y|}{|a-y|^2+(x-z)^2}\bigg)\]
\item
\[P= \frac{1}{2\pi}\int e^{-iwz} dz  e^{iwx }e^{-|a-y||w|}dw \]
\[\hat{P}=  e^{-iwz} dz  e^{-|a-y||w|} \]
\[\hat{P}_t =0\]
\[\hat{P}_{xx}= -w^2\hat{P}\]
\[\hat{P}_{yy}= w^2\hat{P}\]
\[\hat{P}_t + \hat{P}_{xx} + \hat{P}_{yy} =0\]
Thus, this solves the PDE
\[P(t,x,y)= \frac{1}{2\pi}\int e^{-iwz} dz  e^{iwx }e^{-|a-y||w|}dw \]
As t goes to T,
\[P(T,x,a)=\frac{1}{2\pi}\int e^{-iwz} dz  e^{iwx } dw = \mathbb{I}_{dz}\]



\end{enumerate}

\end{enumerate}
\end{document}