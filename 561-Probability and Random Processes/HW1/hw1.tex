\documentclass[a4paper,11pt]{article}

\usepackage{fullpage} % Package to use full page
\usepackage{parskip} % Package to tweak paragraph skipping
\usepackage{amsmath}
\usepackage{hyperref}
\usepackage{amsmath,amsfonts,amsthm} % Math packages
\usepackage{graphicx}
\usepackage{listings}
\usepackage{caption}
\usepackage{subcaption}
\usepackage{color}
\usepackage{float}
\definecolor{codegreen}{rgb}{0,0.6,0}
\definecolor{codegray}{rgb}{0.5,0.5,0.5}
\definecolor{codepurple}{rgb}{0.58,0,0.82}
\definecolor{backcolour}{rgb}{0.95,0.95,0.92}
\definecolor{brown}{rgb}{0.59, 0.29, 0.0}
\definecolor{beaublue}{rgb}{0.74, 0.83, 0.9}
\definecolor{orange}{rgb}{1.0, 0.5, 0.0}
\definecolor{darkslategray}{rgb}{0.18, 0.31, 0.31}
\def\Xint#1{\mathchoice
	{\XXint\displaystyle\textstyle{#1}}%
	{\XXint\textstyle\scriptstyle{#1}}%
	{\XXint\scriptstyle\scriptscriptstyle{#1}}%
	{\XXint\scriptscriptstyle\scriptscriptstyle{#1}}%
	\!\int}
\def\XXint#1#2#3{{\setbox0=\hbox{$#1{#2#3}{\int}$}
		\vcenter{\hbox{$#2#3$}}\kern-.5\wd0}}
\def\dashint{\Xint-}

% Swap the definition of \abs* and \norm*, so that \abs
% and \norm resizes the size of the brackets, and the 
% starred version does not.
\makeatletter
\let\oldabs\abs
\def\abs{\@ifstar{\oldabs}{\oldabs*}}
%
\let\oldnorm\norm
\def\norm{\@ifstar{\oldnorm}{\oldnorm*}}
\makeatother
\definecolor{keywords}{RGB}{255,0,90}
\definecolor{comments}{RGB}{0,0,113}
\definecolor{red}{RGB}{160,0,0}
\definecolor{green}{RGB}{0,150,0}
\definecolor{codegreen}{rgb}{0,0.6,0}
\definecolor{codegray}{rgb}{0.5,0.5,0.5}
\definecolor{codepurple}{rgb}{0.58,0,0.82}
\definecolor{backcolour}{rgb}{0.95,0.95,0.92}
\definecolor{brown}{rgb}{0.59, 0.29, 0.0}
\definecolor{beaublue}{rgb}{0.74, 0.83, 0.9}
\definecolor{orange}{rgb}{1.0, 0.5, 0.0}
\definecolor{darkslategray}{rgb}{0.18, 0.31, 0.31}
\definecolor{deepblue}{rgb}{0,0,0.5}
\definecolor{deepred}{rgb}{0.6,0,0}
\definecolor{deepgreen}{rgb}{0,0.5,0}
\lstdefinestyle{myMatlabstyle}{
	language=Matlab,
	backgroundcolor=\color{white},   
	commentstyle=\color{codegreen},
	keywordstyle=\color{blue},
	identifierstyle=\color{brown},
	numberstyle=\tiny\color{codegray},
	stringstyle=\color{orange},
	basicstyle=\footnotesize,
	breakatwhitespace=false,         
	breaklines=true,                 
	captionpos=b,                    
	keepspaces=true,                 
	numbers=left,                    
	numbersep=5pt,                  
	showspaces=false,                
	showstringspaces=false,
	showtabs=false,                  
	tabsize=2
}
\lstdefinestyle{myPythonstyle}{
	language=Python, 
	basicstyle=\ttfamily\small, 
	keywordstyle=\color{blue},
	commentstyle=\color{green},
	stringstyle=\color{red},
	showstringspaces=false,
	identifierstyle=\color{black},
}
\lstset{language=Matlab,frame=single}
\lstset{language=Python,frame=single}

\title{AMATH 561: Homework 1}
\author{Jithin D. George, No. 1622555}
%\date{23/11/16}
% matrix environment
\newenvironment{mat}{\left[ \begin{array}{ccccccccccccc}}{\end{array}\right]}
\newcommand\bcm{\begin{mat}}
	\newcommand\ecm{\end{mat}}

\begin{document}

\maketitle
\begin{enumerate}

	\item
Setting A as B and as $\phi$, we find that $B,\phi\in \mathcal{G}$.

If $A_1,A_2,A_3,\ldots \in \mathcal{F}$, then $\cup_{i}A_i \in \mathcal{F}$.

Since $\cup_{i}A_i \in \mathcal{F}$.,
\[\cup_{i}(A_i\cap B) = (\cup_{i}A_i)\cap B \text{ exists in } \mathcal{G}.\]

For any A $\in \mathcal{F}$, there exist its complement $A^c$.

So, the following exist.

\[Y_1=A \cap B, Y_2=A^c \cap B\]
The relative complement of $Y_1$ with respect to B is given by
\[B\setminus Y_1 = B\setminus(A \cap B)= (B\setminus B)\cup(B\setminus A)=\phi \cup (B\setminus A)=B\setminus A = B\cap A^c = Y_2\]
So, if Y is an element in \mathcal{G}, then its complement exists as well in \mathcal{G}.



Thus, $\mathcal{F}$ is a $\sigma$ -algebra on B

\item
Since $\mathcal{F}$ and $\mathcal{G}$ are $\sigma$-algebra,

\[ \phi \in \mathcal{F}, \phi \in \mathcal{G}\]

So, \[\phi \in \mathcal{F} \cap \mathcal{G}\]

Also, if A is in $\mathcal{F}\cap \mathcal{G}$, then 
\[A \in \mathcal{F} \text{  and  } A \in \mathcal{G}\]
and
\[A^c \in \mathcal{F} \text{  and  } A^c \in \mathcal{G}\]
Thus, \[A^c \in \mathcal{F} \cap \mathcal{G}\]
Finally, if
\[A_1,A_2,A_3,A_4,\ldots \in \mathcal{F} \cap \mathcal{G} \]
then for countable unions,
\[\cup_i A_i \in \mathcal{F} \cap \mathcal{G}\]
since 
\[A_1,A_2,A_3,A_4,\ldots, \cup_i A_i \in \mathcal{F} \text{  and  }A_1,A_2,A_3,A_4,\ldots, \cup_i A_i \in \mathcal{G}\]
Thus, $\mathcal{F} \cap \mathcal{G}$ is a $\sigma$-algebra.

\item
\begin{enumerate}
 \item Here, $\Omega$ is the set of all possible outcomes and is given by
 \[\Omega =\{TTT,HHH,HTT,HHT,HTH,THH,THT,TTH\}\]
 The trivial $\sigma$ -algebra is given by
 \[\mathcal{F_0}= \{\phi,\Omega\}\] with the probability measure maps them to {0,1}
 A slightly more resolved $\sigma$ -algebra is given by
 \[\mathcal{F}_1= \{\phi,\Omega,HXX ,TXX \}\] where HXX represents the event where the firstcoin shows a Head while
 TXX is its complement where the first coin shows a tail.
 
 If p is the probability of getting a Head, the probability measure maps the $\sigma$ -algebra to \{0,1,p,1-p\} 
 \item
 
 \[\Omega =\{Blue Blue,Red Red,Blue Red, Red Blue\}\]
 
  A slightly resolved $\sigma$ -algebra is given by
 \[\mathcal{F}= \{\phi,\Omega,\text{ `Two balls of same color `} ,\text{ `Two balls of different color `}\}\] 
 
 The probability measure maps this $\sigma$ -algebra to \{0,1,$\frac{1}{3},\frac{1}{3}$\} 
 

 \item
 Here, $\Omega$ is infinite and given by
 \[\Omega =\{H,TH,TTH,TTTH,TTTTH,TTTTTH \ldots\}\]
   A $\sigma$ -algebra is given by
 \[\mathcal{F}= \{\phi,\Omega,\text{ `Events where you get Head within 3 tries`} ,\text{ `Events where you don't get Head within 3 tries`}\}\] 
 
  If p is the probability of getting a Head is p,the probability measure maps this $\sigma$ -algebra to \{0,1,$p+(1-p)p+(1-p)^2p$,1-($p+(1-p)p+(1-p)^2p$)\} 
\end{itemize}

\item
Since g is a strictly increasing function, it is a one-one function. Since it is continuous, it is onto. Thus, g is invertible.

So, we can write
\[F_Y(x)=P(Y\leq x)=P(g(X)\leq x)= P(X\leq g^{-1}(x)) = F_X(g^{-1}(x))\]

By using chain rule, we can get 
\[f_Y(x) = {F'}_Y(x) = {F'}_X(g^{-1}(x))g^{-1}^{'}(x)\]

\item
\begin{enumerate}
 \item 
 \[F_Y(x)=P(Y\leq x)=P(X^2\leq x)= P(-\sqrt{x}\leq X\leq \sqrt{x}) = F_X(\sqrt{x})-F_X(-\sqrt{x})\]
 
 Here, x should be greater than or equal to 0.
 So,
 \begin{align*}
  F_Y = \begin{cases}
      0 & \text{if $x<0$} \\
       F_X(\sqrt{x})-F_X(-\sqrt{x}) & \text{if $x\geq 0$}
    \end{cases}
\end{align*}
 \item 
 
 \[F_Y(x)=P(Y\leq x)=P(\sqrt{|X|}\leq x)= P(|X|\leq x^2)=P(-x^2 \leq X\leq x^2)= F_X(x^2)-F_X(-x^2) \]
 Since usually the square root function is assumed to be its positive value and not multivalued,
 So,
 \begin{align*}
  F_Y = \begin{cases}
      0 & \text{if $x<0$} \\
       F_X(x^2)-F_X(-x^2) & \text{if $x\geq 0$}
    \end{cases}
\end{align*}
\item

\[F_Y(x)=P(Y\leq x)=P(sin(X)\leq x)\]
This is only valid for $-1\leq x \leq 1$.
Within this region, for $sin(X)\leq x$, the random variable X can take any value in \mathcal{R} depending on x.
Also, for a particular x, the corresponding X such that $sin(X)\leq x$ is a collection of infinitely many sets since sin(x) is periodic.

We define

\[ \eta = \cup_k \{-\pi + 2k\pi  \leq X \leq arcsin(x) + 2k\pi\} \forall k \in  \mathcal{Z}  \]

 Then,
 \begin{align*}
  F_Y = \begin{cases}
      0 & \text{if $x<-1$} \\
       \\\int_{\eta} dx F_X^{'}(x) & \text{if $0\leq x\leq 1$}
       \\
       1 & \text{if $x > 1$}
    \end{cases}
\end{align*}
\item
\[F_Y(x)=P(Y\leq x)=P(F_X(X)\leq x)\]
Although $F_X$ is non-decreasing, it may not be continuous and increasing. 
So, it may not be invertible. But, for any x, the random variable X that satisfies
$F_X(X)\leq x)$ by an open or closed set like in the graph below (since it's non-decreasing and
right-continuous). Thus, everything is defined on a Borel $\sigma$-algebra and $F_X$(x) is still a measurable function.
\newpage

Unlike the previous case, here the set $\{X: F_X(X) \leq x\}$ is one single set (since $F_X$ is 
non-decreasing). So, we can find $F_Y$ by evaluating $F_X$ at K such that $K = sup\{X: F_X(X) \leq x\}$.
 \begin{align*}
  F_Y = \begin{cases}
      0 & \text{if $x<0$} \\
       \\\ F_X(sup\{X: F_X(X) \leq x\}) & \text{if $0\leq x\leq 1$}
       \\
       1 & \text{if $x > 1$}
    \end{cases}
\end{align*}

 \end{enumerate}

\item 
\begin{enumerate}
 \item 
We need $\mathbb{E}$Z =1.
By definition,
\[\mathbb{E}Z = \int_R dx Z f(x)= \int_R dx \frac{g(x)}{f(x)} f(x)\]
Since f $>$ 0 always, we don't need to worry about singularities in the integral.
\[\mathbb{E}Z = \int_R dx Z f(x)= \int_R dx  g(x) = \int_{-\infty}^{\infty} g(x)dx\]
That is the distribution function corresponding to g evaluated at infinity, which should be 1. Thus,
\[\mathbb{E}Z =1\]

Also, since g $\geq$ 0 because it is a density function and f $>$ 0, Z $\geq$ 0.

Thus, Z fulfills the requirements to be a Radon-Nikodym derivative.
\item
To find the density of X, we start with the distribution of X (under the new probability measure).
\[\widetilde{\mathcal{P}}(X\leq b) = \mathbb{E}Z\mathbbm{1}_{\{X\leq b\}} = \int_{\{X\leq b\}}  dx Z f(x)= \int_{\{X\leq b\}}  dx  g(x) = \int_{-\infty}^{b} g(x)dx\]
 Looking at this, we can see that the density function of X under the new measure is g(x). 
\end{enumerate} 
	\end{enumerate} 
\end{document}