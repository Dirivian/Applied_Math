\documentclass[a4paper,11pt]{article}

\usepackage{fullpage} % Package to use full page
\usepackage{parskip} % Package to tweak paragraph skipping
\usepackage{amsmath}
\usepackage{hyperref}
\usepackage{amsmath,amsfonts,amsthm} % Math packages
\usepackage{graphicx}
\usepackage{listings}
\usepackage{caption}
\usepackage{subcaption}
\usepackage{color}
\usepackage{float}
\definecolor{codegreen}{rgb}{0,0.6,0}
\definecolor{codegray}{rgb}{0.5,0.5,0.5}
\definecolor{codepurple}{rgb}{0.58,0,0.82}
\definecolor{backcolour}{rgb}{0.95,0.95,0.92}
\definecolor{brown}{rgb}{0.59, 0.29, 0.0}
\definecolor{beaublue}{rgb}{0.74, 0.83, 0.9}
\definecolor{orange}{rgb}{1.0, 0.5, 0.0}
\definecolor{darkslategray}{rgb}{0.18, 0.31, 0.31}
\def\Xint#1{\mathchoice
	{\XXint\displaystyle\textstyle{#1}}%
	{\XXint\textstyle\scriptstyle{#1}}%
	{\XXint\scriptstyle\scriptscriptstyle{#1}}%
	{\XXint\scriptscriptstyle\scriptscriptstyle{#1}}%
	\!\int}
\def\XXint#1#2#3{{\setbox0=\hbox{$#1{#2#3}{\int}$}
		\vcenter{\hbox{$#2#3$}}\kern-.5\wd0}}
\def\dashint{\Xint-}

% Swap the definition of \abs* and \norm*, so that \abs
% and \norm resizes the size of the brackets, and the 
% starred version does not.
\makeatletter
\let\oldabs\abs
\def\abs{\@ifstar{\oldabs}{\oldabs*}}
%
\let\oldnorm\norm
\def\norm{\@ifstar{\oldnorm}{\oldnorm*}}
\makeatother
\definecolor{keywords}{RGB}{255,0,90}
\definecolor{comments}{RGB}{0,0,113}
\definecolor{red}{RGB}{160,0,0}
\definecolor{green}{RGB}{0,150,0}
\definecolor{codegreen}{rgb}{0,0.6,0}
\definecolor{codegray}{rgb}{0.5,0.5,0.5}
\definecolor{codepurple}{rgb}{0.58,0,0.82}
\definecolor{backcolour}{rgb}{0.95,0.95,0.92}
\definecolor{brown}{rgb}{0.59, 0.29, 0.0}
\definecolor{beaublue}{rgb}{0.74, 0.83, 0.9}
\definecolor{orange}{rgb}{1.0, 0.5, 0.0}
\definecolor{darkslategray}{rgb}{0.18, 0.31, 0.31}
\definecolor{deepblue}{rgb}{0,0,0.5}
\definecolor{deepred}{rgb}{0.6,0,0}
\definecolor{deepgreen}{rgb}{0,0.5,0}
\lstdefinestyle{myMatlabstyle}{
	language=Matlab,
	backgroundcolor=\color{white},   
	commentstyle=\color{codegreen},
	keywordstyle=\color{blue},
	identifierstyle=\color{brown},
	numberstyle=\tiny\color{codegray},
	stringstyle=\color{orange},
	basicstyle=\footnotesize,
	breakatwhitespace=false,         
	breaklines=true,                 
	captionpos=b,                    
	keepspaces=true,                 
	numbers=left,                    
	numbersep=5pt,                  
	showspaces=false,                
	showstringspaces=false,
	showtabs=false,                  
	tabsize=2
}
\lstdefinestyle{myPythonstyle}{
	language=Python, 
	basicstyle=\ttfamily\small, 
	keywordstyle=\color{blue},
	commentstyle=\color{green},
	stringstyle=\color{red},
	showstringspaces=false,
	identifierstyle=\color{black},
}
\lstset{language=Matlab,frame=single}
\lstset{language=Python,frame=single}

\title{AMATH 561: Homework 3}
\author{Jithin D. George, No. 1622555}
%\date{23/11/16}
% matrix environment
\newenvironment{mat}{\left[ \begin{array}{ccccccccccccc}}{\end{array}\right]}
\newcommand\bcm{\begin{mat}}
	\newcommand\ecm{\end{mat}}

\begin{document}

\maketitle
\begin{enumerate}


	\item
	We need the probability at each value of X to create the generating function. 
But the probability of the binomial distribution itself is a random variable with a uniform distribution over (0,1). 
Since it is a uniform distribution over (0,1), it has a probability of 1. 	
\[P(X=k)=\int_{0}^{1} P(X=K,U=r)P_U(r)dr\]
\[=\int_{0}^{1} \binom{n}{k} r^k (1-r)^{n-k}1dr\]
\[=\int_{0}^{1} \binom{n}{k} r^k (1-r)^{n-k}dr\]
The generating function is given by
\[G(s)= \sum_k s^kP(X=k)\]
\[= \sum_k s^k\int_{0}^{1} \binom{n}{k} r^k (1-r)^{n-k}dr\]
Assuming uniform convergrence (because these are probability distributions)
\[= \int_{0}^{1} \sum_k  \binom{n}{k} (sr)^k (1-r)^{n-k}dr\]
\[= \int_{0}^{1} (sr+1-r)^{n}dr\]
\[= \frac{ s^{n+1} -1}{(n+1)(s-1)}\]
Luckily, this is the sum of a finite geometric series.
\[G(s) = \frac{1}{n+1}(1+s+s^2+\ldots +s^n)\]
Thus, P(X=k) for any value of k is $\frac{1}{n+1}$
\item
\[\mathbb{E}[Z_n Z_m]=\mathbb{E}[\mathbb{E}[Z_n Z_m|Z_m]]\]
\[=\mathbb{E}[Z_m\mathbb{E}[Z_n|Z_m]]\]
We want to know what $\mathbb{E}[Z_n|Z_m]$ is
\[\mathbb{E}[Z_n|Z_m]=G_{Z_n|Z_m}'(1)\]
\[=G_{Z_{n-1}|Z_m}'(1)G'(1)\]
\[=G_{Z_{n-1}|Z_m}'(1)\mu\]
\[=G_{Z_{m}|Z_m}'(1)\mu^{n-m}\]
\[=\mathbb{E}{Z_{m}|Z_m}\mu^{n-m}\]
\[=Z_{m}\mu^{n-m}\]
Thus,
\[\mathbb{E}[Z_n Z_m]=\mathbb{E}[Z_mZ_{m}\mu^{n-m}\]
\[= \mu^{n-m}\mathbb{E}{Z_m}^2\]

\[\rho(Z_m,Z_n)= \frac{CoV(Z_m,Z_n)}{\sqrt{Var(Z_m) Var(Z_n)}}\]
\[= \frac{\mathbb{E}{Z_mZ_n}-\mathbb{E}{Z_m}\mathbb{E}{Z_n}}{\sqrt{Var(Z_m) Var(Z_n)}}\]

\[= \frac{\mu^{n-m}\mathbb{E}{Z_m}^2-\mu^{n+m}}{\sqrt{Var(Z_m) Var(Z_n)}}\]
\[= \frac{\mu^{n-m}\mathbb{E}{Z_m}^2-\mu^{n-m}\mu^{2m}}{\sqrt{Var(Z_m) Var(Z_n)}}\]
\[= \frac{\mu^{n-m}\mathbb{E}{Z_m}^2-\mu^{n-m}(\mathbb{E}{Z_m})^2}{\sqrt{Var(Z_m) Var(Z_n)}}\]
\[= \mu^{n-m} \frac{\sqrt{Var(Z_m)}}{ \sqrt{Var(Z_n)}}\]

If $\mu = 1, Var(Z_k)=kVar(Z_1) $. Then,
\[\rho(Z_m,Z_n)=\mu^{n-m} \sqrt{\frac{m}{n}} \]
Otherwise, 
\[Var(Z_k)= Var(Z_1) \frac{(\mu^k-1)\mu^{k-1}}{\mu-1}\]
Then,
\[\rho(Z_m,Z_n)=\mu^{n-m} \sqrt{\frac{(\mu^m-1)\mu^{m-1}}{(\mu^n-1)\mu^{n-1}}} \]
\[\rho(Z_m,Z_n)=\mu^{\frac{n-m}{2}} \sqrt{\frac{\mu^m-1}{\mu^n-1}} \]

\item
	Not required
	\item
\[G_{Z_{n+1}}(s)= \mathbb{E}[s^{Z_{n+1}}]\]
\[ = \mathbb{E}[s^{\sum_{i=1}^{Z_n} X_{n,i}+Y_n}]\]
\[ = \mathbb{E}[s^{\sum_{i=1}^{Z_n} X_{n,i}}s^{Y_n}]\]
Since the Xs and Ys are independent,
\[ = \mathbb{E}[s^{\sum_{i=1}^{Z_n} X_{n,i}}]\mathbb{E}[s^{Y_n}]\]
\[ = \mathbb{E}[s^{\sum_{i=1}^{Z_n} X_{n,i}}]G_Y(s)\]
By definition, from Theorem 3.1.9, this is
\[ = G_{Z_n}(G_X(s))G_Y(s)\]
\[G_{Z_{n+1}}(s) = G_{Z_n}(G_X(s))G_Y(s)\]
\[G_{Z_{1}}(s) = G_{Z_0}(G_X(s))G_Y(s)\]
\[ = \mathbb{E}[s^{\sum_{i=1}^{1} X_{n,i}}]G_Y(s)\]
\[ = G_X(s)G_Y(s)\]
\[G_{Z_{2}}(s)= G_{Z_1}(G_X(s))G_Y(s)\]
\[= G_X(G_X(s))G_Y(G_X(s))G_Y(s)\]
\[= (G_X(s))^2G_Y(G_X(s))G_Y(s)\]
\item
\[\phi_{X^2}(t)=\mathbb{E}e^{itX^2}\]
\[=\int_{-\infty}^{\infty}\frac{1}{\sqrt{2\pi \sigma^2}}e^{itx^2}e^{-\frac{(x-\mu)^2}{2\sigma^2}}\]
\[=\int_{-\infty}^{\infty}\frac{1}{\sqrt{2\pi \sigma^2}}e^{itx^2-\frac{(x-\mu)^2}{2\sigma^2}}\]
\[=\int_{-\infty}^{\infty}\frac{1}{\sqrt{2\pi \sigma^2}}e^{-\frac{(1-2\sigma^2it)x^2-2\mu x+\mu^2}{2\sigma^2}}\]
\[=\int_{-\infty}^{\infty}\frac{1}{\sqrt{2\pi \sigma^2}}e^{-\frac{(1-2\sigma^2it)}{2\sigma^2}(x^2-\frac{2\mu x}{1-2\sigma^2it}+\frac{\mu^2}{1-2\sigma^2it})}\]
\[=\int_{-\infty}^{\infty}\frac{1}{\sqrt{2\pi \sigma^2}}e^{-\frac{(1-2\sigma^2it)}{2\sigma^2}((x-\frac{\mu }{1-2\sigma^2it})^2-\frac{\mu^2}{(1-2\sigma^2it)^2}+\frac{\mu^2}{1-2\sigma^2it})}\]
\[=e^{\frac{- it}{(1-2\sigma^2it)}}\int_{-\infty}^{\infty}\frac{1}{\sqrt{2\pi \sigma^2}}e^{-\frac{(1-2\sigma^2it)}{2\sigma^2}(x-\frac{2\mu }{1-it})^2}\]
\[=e^{\frac{-it}{(1-2\sigma^2it)}}\frac{1}{\sqrt{1-2\sigma^2it}}\]
\[\phi_{X^2}(t)=\frac{1}{\sqrt{1-2\sigma^2it}} e^{\frac{-it}{(1-2\sigma^2it)}}\]
In this case, we assume it is possible to analytically continue the complex function.
\item
\begin{enumerate}
 \item 


\[F_{X_n}(x)=(x-\frac{sin(2n\pi x)}{2 n \pi})\mathbb{I}_{0 \leq x\leq1}+ \mathbb{I}_{x>1}\]
This is 0 when x $\leq$ 0 and 1 when $x>0$. Also, when x is in between 0 and 1, this is a strictly increasing function.
Thus, this is a distribution function.

It is smooth between 0 and 1. So, we can differentiate it
 \begin{align*}
  f(x)= \begin{cases}
       1 - cos(2n\pi x) & \text{if $0 \leq x \leq 1$}\\
      \\ 0 & \text{otherwise}
    \end{cases}
\end{align*}
\item
\[\lim_{n\to\infty} F(x)=x\]
This is the uniform distribution function.
\[\lim_{n\to\infty} f(x)= 1-\lim_{n\to\infty}cos(2n\pi x)\]
This limit doesn't exist. It implies that in that limit, the cos term is wildly oscillating and is not the constant 
uniform density function.
\end{enumerate}
\item 

When k = 1, this process is a geomtric distribution where you get tails N-1 times and then heads.
When k = 2, this is like one geometric distribution after another one and thus getting two heads. These two geometric distributions 
are independent of each other.

Thus, for any k, the process is a sequence of k geometric distribution.
So, the generating function is $G_{Geo}(s)^k$, where $G_{Geo}(s)$ is the generating function of the geometric distribution.
Since the characteristic function is a generating function too, we can say the characteristic function of this process is
\[\phi_N(t)= (\phi_{geo}(t))^k\]
\[= (e^{it}p+e^{2it}qp+e^{3it}q^{2}p+ \ldots )^k\]
\[= (\frac{e^{it}p}{1-e^{it}(1-p)})^k\]
\[\phi_{2Np}(t)= \bigg(\frac{e^{2itp}p}{1-e^{itp}(1-p)}\bigg)^k\]
\[\lim_{p\to 0}\phi_{2Np}(t)= \bigg(\frac{1}{1-it}\bigg)^k\]
Characteristic function for a gamma distribution is given by
\[\phi(t) =\int_0^\infty \frac{1}{\Gamma(s)} \lambda^s x^{s-1} e^{-\lambda x +itx} dx \]
Using a result from complex analysis, we know
\[ =\bigg(\frac{\lambda}{\lambda-it}\bigg)^s\]

For $\lambda=1$ and s=k, these characteristic function converges to that of a gamma distribution as p goes to zero.

Thus, by the corollary of the Continuity Theorem, the distribution function for the process converges to a Gamma distribution.
\end{enumerate} 
\end{document}