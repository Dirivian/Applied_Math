\documentclass[a4paper]{article}

\usepackage{fullpage} % Package to use full page
\usepackage{parskip} % Package to tweak paragraph skipping
\usepackage{amsmath}
\usepackage{hyperref}
\usepackage{amsmath,amsfonts,amsthm} % Math packages
\usepackage{graphicx}
\usepackage{listings}
\usepackage{color}
\usepackage{float}
\definecolor{codegreen}{rgb}{0,0.6,0}
\definecolor{codegray}{rgb}{0.5,0.5,0.5}
\definecolor{codepurple}{rgb}{0.58,0,0.82}
\definecolor{backcolour}{rgb}{0.95,0.95,0.92}
\definecolor{brown}{rgb}{0.59, 0.29, 0.0}
\definecolor{beaublue}{rgb}{0.74, 0.83, 0.9}
\definecolor{orange}{rgb}{1.0, 0.5, 0.0}
\definecolor{darkslategray}{rgb}{0.18, 0.31, 0.31}
\def\Xint#1{\mathchoice
	{\XXint\displaystyle\textstyle{#1}}%
	{\XXint\textstyle\scriptstyle{#1}}%
	{\XXint\scriptstyle\scriptscriptstyle{#1}}%
	{\XXint\scriptscriptstyle\scriptscriptstyle{#1}}%
	\!\int}
\def\XXint#1#2#3{{\setbox0=\hbox{$#1{#2#3}{\int}$}
		\vcenter{\hbox{$#2#3$}}\kern-.5\wd0}}
\def\dashint{\Xint-}

% Swap the definition of \abs* and \norm*, so that \abs
% and \norm resizes the size of the brackets, and the 
% starred version does not.
\makeatletter
\let\oldabs\abs
\def\abs{\@ifstar{\oldabs}{\oldabs*}}
%
\let\oldnorm\norm
\def\norm{\@ifstar{\oldnorm}{\oldnorm*}}
\makeatother
\lstdefinestyle{mystyle}{
	backgroundcolor=\color{white},   
	commentstyle=\color{codegreen},
	keywordstyle=\color{blue},
	identifierstyle=\color{brown},
	numberstyle=\tiny\color{codegray},
	stringstyle=\color{orange},
	basicstyle=\footnotesize,
	breakatwhitespace=false,         
	breaklines=true,                 
	captionpos=b,                    
	keepspaces=true,                 
	numbers=left,                    
	numbersep=5pt,                  
	showspaces=false,                
	showstringspaces=false,
	showtabs=false,                  
	tabsize=2
}
\lstset{style=mystyle}
\newenvironment{mat}{\left[ \begin{array}{ccccccccccccc}}{\end{array}\right]}

\newcommand\bcm{\begin{mat}}
\newcommand\ecm{\end{mat}}

\title{AMATH 568: Problem Set 6}
\author{Jithin D. George, No. 1622555}
%\date{23/11/16}

\begin{document}

\maketitle
\begin{enumerate}

	
\item 
\begin{enumerate}
	
	\item
Using Laplace's method, since the maximum of cos(t) is at t=0	
\[ \int_{0}^{\pi/4} e^{xcost}cos(nt)dt =  \int_{0}^{\epsilon} e^{xcost}cos(0)dt \]
\[ =  \int_{0}^{\epsilon} e^{xcost}dt \]
Near 0, we can replace cos(t) by its taylor series.
\[ =  \int_{0}^{\epsilon} e^{x(1- t^2/2 + \ldots)}dt \]
\[ =  \int_{0}^{\epsilon} e^{x}e^{- xt^2/2}e^{ xt^4/4! +\ldots}  dt \]
\[ =  \int_{0}^{\epsilon} e^{x}e^{- xt^2/2}(1+(xt^4/4!-xt^6/6! +\ldots ) +(xt^4/4!-xt^6/6! +\ldots )^2) \ dt \]
\[ = e^{x} \int_{0}^{\epsilon} e^{- xt^2/2}(1+(xt^4/4!-xt^6/6! +\ldots ) +(xt^4/4!-xt^6/6! +\ldots )^2) \ dt \]
Using Laplace's method again, 
\[ =  e^{x} \int_{0}^{ \infty }e^{- xt^2/2}(1+(xt^4/4!-xt^6/6! +\ldots ) +(xt^4/4!-xt^6/6! +\ldots )^2) \ dt\]
\[ =  e^{x} \sqrt{\frac{\pi}{2x}} + e^{x} x  \sqrt{\frac{9 (4!)^5 \pi}{16 x^5}}+ \ldots\]
\[ =  e^{x} \sqrt{\frac{\pi}{2x}} + O(\frac{1}{\sqrt{x^3}})\]
	\item
	cosh(t) has a minimum at t =0 and it is an even function. Using Laplace's method, 
	\[ \int_{-1}^{1} e^{-xcosht}dt =\int_{-\epsilon}^{\epsilon} e^{-xcosht}dt  \] 
\[ = 2 \int_{0}^{\epsilon} e^{-xcosht}dt  \] 
\[ = 2 \int_{0}^{\epsilon} e^{-x(1+ t^2/2 + \ldots)}dt  \]
\[ = 2 e^{-x}\int_{0}^{\epsilon} e^{-x( t^2/2 + \ldots)}dt  \] 	 
\[ = 2 e^{-x}\int_{0}^{\epsilon} e^{-x( t^2/2 + \ldots)}dt  \] 
\[ =  2 e^{-x}\int_{0}^{ \infty }e^{- xt^2/2}(1+(xt^4/4!+xt^6/6! +\ldots ) +(xt^4/4!-xt^6/6! +\ldots )^2) \ dt\]
\[ = 2 e^{-x}\sqrt{\frac{\pi}{2x}} + 2e^{-x} x  \sqrt{\frac{9 (4!)^5 \pi}{16 x^5}}+ \ldots\]
\[ =2 e^{-x} \sqrt{\frac{\pi}{2x}} + O(\frac{1}{\sqrt{x^3}})\]	
		\item
		
		\[ \int_{0}^{1} t^{x}cos^n(\pi t)dt  =\int_{0}^{1} e^{xlog(t)}cos^n(\pi t)dt \]
The integral is valid only near t=1.
\[ =\int_{0}^{1} t^x ( -1 + \frac{1}{2}\pi^2(t -1)^2+\ldots)^n dt \]
Using Laplace's method,
\[ =\int_{1-\epsilon}^{1} t^x ( -1 + \frac{1}{2}\pi^2(\epsilon)^2+\ldots)^n dt \]
\[ =\int_{1-\epsilon}^{1} t^x ( -1 + \frac{1}{2}\pi^2(\epsilon)^2+\ldots)^n dt \]
As $\epsilon$ is determined by x, 
\[ =\int_{0}^{1} t^x ( (-1)^n + \frac{1}{2}\pi^2(\frac{1}{x})^2+\ldots) dt \]
\[ =  \Big[\frac{t^{x+1} (-1)^n}{x+1} \Big]_0^1   + \Big[\frac{t^{x+1} (-1)^n}{x+1} \Big]_0^1\frac{1}{2}\pi^2(\frac{1}{x})^2+\ldots  \]
\[ =  \frac{ (-1)^n}{x+1}    + O(\frac{1}{x^3})+\ldots  \]

\end{enumerate}	
	    	
\item
\begin{enumerate}
	\item 
			\[ \int_{-\infty}^{\infty} e^{-x(t-u)^2}g(t)dt = \int_{-\infty}^{u} e^{-x(t-u)^2}g(t)dt + \int_{u}^{\infty} e^{-x(t-u)^2}g(t)dt \]  
\[ = \int_{-\infty}^{u} e^{-x(t-u)^2}g(t)dt + \int_{u}^{\infty} e^{-x(t-u)^2}g(t)dt \] 
\[ = \int_{-\infty}^{u} e^{-x(t-u)^2}g(t)dt + \int_{u}^{\infty} e^{-x(t-u)^2}g(t)dt \] 
\[ =  - \int_{0}^{\infty} e^{-xs}g(u-\sqrt{s})\frac{ds}{2\sqrt{s}} + \int_{0}^{\infty} e^{-xs}g(u+\sqrt{s})\frac{ds}{2\sqrt{s}} \]  

\[ =   \int_{0}^{\infty} e^{-xs}\frac{g(u+\sqrt{s})-g(u-\sqrt{s})ds}{2\sqrt{s}} \]  
\[g(u+\sqrt{s}) = g(u)+  \sqrt{s}g'(u)+ \sqrt{s}g''(u) +\ldots\]
\[g(u-\sqrt{s}) = g(u)- \sqrt{s}g'(u)+ \sqrt{s}g''(u) +\ldots\]
\[g(u+\sqrt{s}) -g(u-\sqrt{s}) = 2\sqrt{s^3}g'(u)+ 2sg'''(u) +\ldots\]
\[ =   \int_{0}^{\infty} e^{-xs}(g'(u)+ sg'''(u) +\ldots ) \]

\[ =   \int_{0}^{\infty} e^{-xs}\sum_{k=0}^{\infty} g^{2k+1}(u)s^k ) \]	
\[ =  \sum_{k=0}^{\infty} \frac{g^{2k+1}(u)}{2k+1} \int_{0}^{\infty} e^{-xs}s^k  \]
Using v=xs,

\[ =  \sum_{k=0}^{\infty}\frac{ g^{2k+1}(u)}{2k+1} \int_{0}^{\infty} e^{-v}(\frac{v}{x})^k  \]
\[ =  \sum_{k=0}^{\infty} \frac{g^{2k+1}(u)}{2k+1}  \frac{\Gamma(k+1)}{x^{k+1}}   \]		
	\item 
	\[ \int_{-\infty}^{\infty} e^{-x(t-u)^2}  \sum_{0}^{\infty} \frac{g^k(u)}{k!}(t-u)^k dt =  \int_{-\infty}^{\infty} e^{-xs^2}  \sum_{0}^{\infty} \frac{g^k(u)}{k!}(s)^k ds\]
		\[  =  \sum_{0}^{\infty} \frac{g^k(u)}{k!} \int_{-\infty}^{\infty} e^{-xs^2}  (s)^k ds \]

Since the odd moments are odd functions,
				\[  =  \sum_{0}^{\infty} \frac{g^{2k}(u)}{2k!} \int_{-\infty}^{\infty} e^{-xs^2}  (s)^{2k} ds\]
				\[  =  \sum_{0}^{\infty} \frac{g^{2k}(u)}{2k!} \frac{(2k-1)!!}{x^k 2^k} \sqrt{\frac{\pi}{x}} \]	
Using an identity from online,							
				\[  =  \sum_{0}^{\infty} \frac{g^{2k}(u)}{2k!} \frac{\Gamma(k+1/2)}{x^{k+1/2} }  \]					
\end{enumerate}

\item

	\[ \int_{0}^{\infty} e^{-xt^3}g(t)dt = \int_{0}^{\infty} e^{-xt^3}(g(0) + g'(0)t +\ldots )dt\]
	\[  = \sum_{k=0}^{\infty} \frac{g^k(0)}{k!}\int_{0}^{\infty} e^{-xt^3}t^k dt\]
Using a substitution,
		\[  = \sum_{k=0}^{\infty}\frac{1}{3} \frac{g^k(0)}{k!}\int_{0}^{\infty} e^{-xv} v^{k/3} v^{-2/3} dv\]
				\[  = \frac{1}{3}\sum_{k=0}^{\infty} \frac{g^k(0)}{k!}\int_{0}^{\infty} e^{-xv} v^{(k-2)/3}  dv\]
Applying Watson's Lemma,
				\[  = \frac{1}{3}\sum_{k=0}^{\infty} \frac{g^k(0) \Gamma(k/3 +1/3) }{k!x^{k/3+ 1/3 }}  \]
				\[  = \frac{g(0) \Gamma(k/3 +1/3 ) }{3 k! x^{1/3}} + O(\frac{1}{x^{2/3}})   \]				
\item 
\[\int_0^1 dy \int_0^y \exp(\frac{\phi(y)-\phi(x)}{\epsilon}) dx \]
\[=\int_0^1 dy \exp(\frac{\phi(y)}{\epsilon})\underbrace{\int_0^y \exp(\frac{-\phi(x)}{\epsilon}) dx }_{g(y)}\]
\[=\int_0^1 dy \exp(\frac{\phi(y)}{\epsilon})g(y)\]
Using Laplace's method and the maxima at $x_2$,
\[=\int_{x_2-\epsilon}^{x_2+\epsilon} dy \exp(\frac{\phi(x_2) +\phi''(x_2)(x-x_2)^2 +\ldots }{\epsilon})g(x_2)\]
\[=\int_{x_2-\epsilon}^{x_2+\epsilon}  e^{\phi(x_2)}e^{\phi''(x_2)(x-x_2)^2 }\exp(\frac{\phi'''(x_2)(x-x_2)^3 +\phi''''(x_2)(x-x_2)^4 +\ldots }{\epsilon})g(x_2) dy\]
\[=\int_{x_2-\epsilon}^{x_2+\epsilon}  e^{\frac{\phi(x_2)}{\epsilon}}e^{\frac{\phi''(x_2)(x-x_2)^2 }{\epsilon}}[1+ (\frac{\phi'''(x_2)(x-x_2)^3 +\phi''''(x_2)(x-x_2)^4 +\ldots }{\epsilon})  \]
\[+ (\frac{\phi'''(x_2)(x-x_2)^3 +\phi''''(x_2)(x-x_2)^4 +\ldots }{\epsilon})^2  +\ldots ]g(x_2) dy\ \]
\[=\int_{x_2-\epsilon}^{x_2+\epsilon} e^{\frac{\phi(x_2)}{\epsilon}}e^{\frac{\phi''(x_2)(x-x_2)^2 }{\epsilon}}[1+ (\frac{\phi'''(x_2)(x-x_2)^3 +\phi''''(x_2)(x-x_2)^4 +\ldots }{\epsilon})  \]
\[+ (\frac{\phi'''(x_2)(x-x_2)^3 +\phi''''(x_2)(x-x_2)^4 +\ldots }{\epsilon})^2  +\ldots ]g(x_2) dy\ \]
\[= e^{\frac{\phi(x_2)}{\epsilon}} g(x_2) \int_{x_2-\epsilon}^{x_2+\epsilon} e^{\frac{\phi''(x_2)(x-x_2)^2 }{\epsilon}}[1+ (\frac{ \phi''''(x_2)(x-x_2)^4  }{\epsilon}) +\ldots ] dy\ \]

\[= e^{\frac{\phi(x_2)}{\epsilon}} g(x_2) \int_{-\infty}^{\infty} e^{\frac{\phi''(x_2)(x-x_2)^2 }{\epsilon}}[1+ (\frac{ \phi''''(x_2)(x-x_2)^4  }{\epsilon}) +\ldots ] dy\ \]
\[= e^{\frac{\phi(x_2)}{\epsilon}} g(x_2) \sqrt{\frac{2\pi \epsilon}{\phi''(x_2)}} + O(\epsilon^{3/2})\ \]

\[ g(x_2) = \int_0^{x_2} \exp(\frac{-\phi(x)}{\epsilon}) dx\\ \]


This uses the minimum at $x_1$ which is before $x_2$ and like before, we can find
\[ g(x_2) = e^{\frac{-\phi(x_1)}{\epsilon}}  \sqrt{\frac{2\pi \epsilon}{\phi''(x_1)}}\\ \]

Thus, 
\[\int_0^1 dy \int_0^y \exp(\frac{\phi(y)-\phi(x)}{\epsilon}) dx  =  e^{\frac{\phi(x_2)-\phi(x_1)}{\epsilon}}  \frac{2\pi \epsilon}{\sqrt{\phi''(x_1)\phi''(x_2)}}\]
\end{enumerate}


\end{document}