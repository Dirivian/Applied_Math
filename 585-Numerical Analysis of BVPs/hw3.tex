%  This is a LaTex file.

%  Homework for the course "AMath 585:  Applied Linear Algebra and Numerical Analysis", 
%  Autumn quarter, 2009, Anne Greenbaum.


%   A latex format for making homework assignments.


\documentclass[letterpaper,12pt]{article}

%          The page format, somewhat wider and taller page than in art12.sty.

\topmargin -0.1in \headsep 0in \textheight 8.9in \footskip 0.6in
\oddsidemargin 0in  \evensidemargin 0in  \textwidth 6.5in
\usepackage{graphicx}
\usepackage{listings}
\usepackage{caption}
\usepackage{subcaption}
\usepackage{color}
\usepackage{float}
\definecolor{keywords}{RGB}{255,0,90}
\definecolor{comments}{RGB}{0,0,113}
\definecolor{red}{RGB}{160,0,0}
\definecolor{green}{RGB}{0,150,0}
\definecolor{codegreen}{rgb}{0,0.6,0}
\definecolor{codegray}{rgb}{0.5,0.5,0.5}
\definecolor{codepurple}{rgb}{0.58,0,0.82}
\definecolor{backcolour}{rgb}{0.95,0.95,0.92}
\definecolor{brown}{rgb}{0.59, 0.29, 0.0}
\definecolor{beaublue}{rgb}{0.74, 0.83, 0.9}
\definecolor{orange}{rgb}{1.0, 0.5, 0.0}
\definecolor{darkslategray}{rgb}{0.18, 0.31, 0.31}
\definecolor{deepblue}{rgb}{0,0,0.5}
\definecolor{deepred}{rgb}{0.6,0,0}
\definecolor{deepgreen}{rgb}{0,0.5,0}
\lstdefinestyle{myMatlabstyle}{
	language=Matlab,
	backgroundcolor=\color{white},   
	commentstyle=\color{codegreen},
	keywordstyle=\color{blue},
	identifierstyle=\color{brown},
	numberstyle=\tiny\color{codegray},
	stringstyle=\color{orange},
	basicstyle=\footnotesize,
	breakatwhitespace=false,         
	breaklines=true,                 
	captionpos=b,                    
	keepspaces=true,                 
	numbers=left,                    
	numbersep=5pt,                  
	showspaces=false,                
	showstringspaces=false,
	showtabs=false,                  
	tabsize=2
}
\lstdefinestyle{myPythonstyle}{
	language=Python, 
	basicstyle=\ttfamily\small, 
	keywordstyle=\color{blue},
	commentstyle=\color{green},
	stringstyle=\color{red},
	showstringspaces=false,
	identifierstyle=\color{black},
}
\lstset{language=Matlab,frame=single}
\lstset{language=Python,frame=single}
\usepackage{epsfig}         % to insert PostScript figures

\begin{document}


%          Definitions of commonly used symbols.



%          The title and header.

\noindent
{\scriptsize AMath 585, Winter 2018} \hfill 

\begin{center}
\large
Assignment 1. 
\normalsize

Jithin D. George, No. 1622555

\end{center}


\noindent
Due Friday, Jan. 12.
\vspace{.3in}

%           The questions!



\noindent
Reading:  Ch 1. Secs. 2.1-2.9.
\vspace{.5cm}

\begin{enumerate}
\item
Use MATLAB to evaluate the second order accurate approximation 
\[
u'' (x) \approx \frac{u(x+h) + u(x-h) - 2 u(x)}{h^2}
\]
for $u(x) = \sin x$ and $x = \pi / 6$.  Try $h = 10^{-1} , 10^{-2} , \ldots ,
10^{-16}$, and make a table of values of $h$, the computed finite difference
quotient, and the error.  Explain your results.


{\bf Solution:}

	
\begin{tabular}{rrr}
\hline
      h &   Computed Value &         Error \\
\hline
 0.1    &     -0.499583    &   0.000416528 \\
 0.01   &     -0.499996    &   4.16665e-06 \\
 0.001  &     -0.5         &   4.15635e-08 \\
 0.0001 &     -0.5         &   3.03874e-09 \\
 1e-05  &     -0.500001    &  -1.15159e-06 \\
 1e-06  &     -0.499933    &   6.6572e-05  \\
 1e-07  &     -0.4996      &   0.000399639 \\
 1e-08  &     -1.11022     &  -0.610223    \\
 1e-09  &    111.022       & 111.522       \\
 1e-10  &      0           &   0.5         \\
 1e-11  &      0           &   0.5         \\
 1e-12  &      1.11022e+08 &   1.11022e+08 \\
 1e-13  &      1.11022e+10 &   1.11022e+10 \\
 1e-14  &     -1.11022e+12 &  -1.11022e+12 \\
 1e-15  &      0           &   0.5         \\
 1e-16  &     -1.11022e+16 &  -1.11022e+16 \\
\hline
\end{tabular}


	\begin{lstlisting}[style=MyPythonstyle]
import numpy as np
from tabulate import tabulate

def central_diff(f,x,h):
    return (f(x+h)+f(x-h)-2*f(x))/h**2
j=[]
hlist =[]
for i in range(1,17):
    hlist.append(10**-i)
for h in hlist:
    j.append([h,central_diff(np.sin,np.pi/6,h),central_diff(np.sin,np.pi/6,h)+np.sin(np.pi/6)])

print(tabulate(j,headers=['h', 'Computed Value', 'Error'],tablefmt="latex"))
	\end{lstlisting}
	
Truncation error($\frac{\sqrt{3}h^2}{48}$) and rounding error($\frac{10^{-16}}{h^2}$) are comparable when $h^4=27*10^{-16}$ i.e $h\approx 10^{-4}$. At $10^{-5}$, we see the first loss of accuracy as step size decreases. This would be where the rounding error begins to dominate truncation error.
Then, at $10^{-8}$, $h^2$ reaches machine precision and computed value gets very wrong because of underflow (the value is too small) or overflow (the denominator is too small leading to a large value overall)
%--------------------------------------------------------------------------


\item
Use the formula in the previous exercise with $h=0.2$, $h=0.1$, and $h=0.05$ 
to approximate $u'' (x)$, where $u(x) = \sin x$ and $x = \pi / 6$.  Use one 
step of Richardson extrapolation, combining the results from $h=0.2$ and $h=0.1$, 
to obtain a higher order accurate approximation.  Do the same with the results 
from $h=0.1$ and $h=0.05$.  Finally do a second step of Richardson extrapolation, 
combining the two previously extrapolated values, to obtain a still higher 
order accurate approximation.  Make a table of the computed results and their 
errors.  What do you think is the order of accuracy after one step of Richardson 
extrapolation?  How about after two?


{\bf Solution:}

\[u(x+h) = u(x) +hu'(x) +\frac{1}{2}h^2u''(x) +O(h^3) \]
\[
u'' (x) \approx \frac{u(x+h) + u(x-h) - 2 u(x)}{h^2} +\frac{1}{12}h^2u''''(x) + C_1 h^4u''''''(x) + O(h^6)
\]
When $h_1$ = h/2,
\[
u'' (x) \approx 4\frac{u(x+h_1) + u(x-h_1) - 2 u(x)}{h^2} +\frac{1}{48}h^2u''''(x) + \frac{1}{4}C_1 h^4u''''''(x) + O(h_1^6)
\]
The error is O($h^2$) right now. After the first iteration, we could get rid of one term and make it O($h^4$) and after a second iteration, make it O($h^6$).

	\begin{lstlisting}[style=MyPythonstyle]
r=[]
s=[]
def richardson(A,f,x,h,step):
    k=2
    if step ==1:
        return (2**k *A(f,x,h/2)-A(f,x,h))/(2**k -1)
    else:
        step-=1
        k+=2
        return (2**k *richardson(A,f,x,h/2,step)-richardson(A,f,x,h,step))/(2**k -1)
for h in [0.2,0.1]:
    r.append([h,richardson(central_diff,np.sin,np.pi/6,h,1),richardson(central_diff,np.sin,np.pi/6,h,1)+np.sin(np.pi/6)])

for h in [0.2,0.1]:
    s.append([h,richardson(central_diff,np.sin,np.pi/6,h,2),richardson(central_diff,np.sin,np.pi/6,h,2)+np.sin(np.pi/6)])
print(tabulate(r,headers=['h', 'Computed Value', 'Error'],tablefmt="latex"))
print(tabulate(s,headers=['h', 'Computed Value', 'Error'],tablefmt="latex"))
	\end{lstlisting}

\begin{tabular}{rrr}
\hline
   h &   Computed Value &       Error \\
\hline
 0.2 &        -0.499999 & 5.5506e-07  \\
 0.1 &        -0.5      & 3.47145e-08 \\
\hline
\end{tabular}

Dividing the two errors givs us 15.99 $\approx$ 16 = $2^4$, agreeing with our O($h^4$) estimate.


\begin{tabular}{rrr}
\hline
   h &   Computed Value &       Error \\
\hline
 0.2 &             -0.5 & 2.48054e-11 \\
 0.1 &             -0.5 & 4.70901e-13 \\
\hline
\end{tabular}

Dividing the two errors givs us 53 $\approx 2^6$, agreeing with our O($h^6$) estimate.

\item
Using Taylor series, derive the error term for the approximation
\[
u' (x) \approx \frac{1}{2h} [ -3 u(x) + 4 u(x+h) - u(x+2h) ] .
\]
{\bf Solution:}

%\[u(x+h) = u(x) +hu'(x) +\frac{1}{2}h^2u''(x) +O(h^3) \]
\[4u(x+h) = 4u(x) +4hu'(x) +4\frac{1}{2}h^2u''(x) +4\frac{1}{6}h^3u'''(x)+ O(h^4) \]
\[u(x+2h) = u(x) +2hu'(x) +4\frac{1}{2}h^2u''(x) +8\frac{1}{6}h^3u'''(x)+ O(h^4) \]
Subtracting the two equations, we get
\[
u' (x) = \frac{1}{2h} [ -3 u(x) + 4 u(x+h) - u(x+2h)] + 4 \frac{1}{6}h^3u'''(x)+ O(h^4)]\]  
\[
u' (x) = \frac{1}{2h} [ -3 u(x) + 4 u(x+h) - u(x+2h)] +  \frac{1}{3}h^2u'''(x)+ O(h^3)\]  
Thus, the error term is $  \frac{1}{3}h^2u'''(x)+ O(h^3)$ which is second order.
\item
Consider a forward difference approximation for the second derivative
of the form
\[
u'' (x) \approx A u(x) + B u(x+h) + C u(x+2h) .
\]
Use Taylor's theorem to determine the coefficients $A$, $B$, and $C$
that give the maximal order of accuracy and determine what this order is.

{\bf Solution:}

\[Bu(x+h) = Bu(x) +Bhu'(x) +B\frac{1}{2}h^2u''(x) +B\frac{1}{6}h^3u'''(x)+ BO(h^4) \]
\[Cu(x+2h) = Cu(x) +2Chu'(x) +4C\frac{1}{2}h^2u''(x) +8C\frac{1}{6}h^3u'''(x)+ CO(h^4) \]

Thus, we can easily see that to get rid of u(x) and u'(x), we need
\[A=-B-C, B=-2C\]
But we cannot get rid of u'''(x) term.
\[
u'' (x) \approx  C h^2u''(x) +Ch^3u'''(x)+ (B+C)O(h^4)
\]
Thus, C = $\frac{1}{h^2}$

So,
\[
u'' (x) \approx  u''(x) + hu'''(x)+ O(h^2)
\]
This is first order.
\[C = \frac{1}{h^2}, B = -\frac{2}{h^2}, A = \frac{1}{h^2}\]
\item
Consider the two-point boundary value problem
\[ 
u'' + 2xu' - x^2 u = x^2 ,~~~u(0)=1,~~~u(1) = 0 .
\]
Let $h=1/4$ and explicitly write out the difference equations,
using centered differences for all derivatives.

{\bf Solution:}

Since h = 0.25, there are 3 interior points $u_1,u_2,u_3$ and the boundaries are $u_0$ and $u_4$.

\[\frac{u_2-2u_1+u_0}{(0.25)^2} +2*0.25 \frac{u_2-u_0}{2*0.25}-(0.25)^2u_1 = (0.25)^2\]
\[\frac{u_3-2u_2+u_1}{(0.25)^2} +2*0.5 \frac{u_3-u_1}{2*0.25}-(0.5)^2u_2 = (0.5)^2\]
\[\frac{u_4-2u_3+u_2}{(0.25)^2} +2*0.75 \frac{u_4-u_2}{2*0.25}-(0.75)^2u_3 = (0.75)^2\]
Plugging in the boundary values, this becomes
\[\frac{u_2-2u_1+1}{(0.25)^2} +2*0.25 \frac{u_2-1}{2*0.25}-(0.25)^2u_1 = (0.25)^2\]
\[\frac{u_3-2u_2+u_1}{(0.25)^2} +2*0.5 \frac{u_3-u_1}{2*0.25}-(0.5)^2u_2 = (0.5)^2\]
\[\frac{-2u_3+u_2}{(0.25)^2} +2*0.75 \frac{-u_2}{2*0.25}-(0.75)^2u_3 = (0.75)^2\]
\item
A rod of length 1 meter has a heat source applied to it and it eventually
reaches a steady-state where the temperature is not changing.  The
conductivity of the rod is a function of position $x$ and is given by
$c(x) = 1 + x^2$.  The left end of the rod is held at a constant
temperature of 1 degree.  The right end of the rod is insulated so
that no heat flows in or out from that end of the rod.  This problem is
described by the boundary value problem:
\[
\frac{d}{dx} \left( (1 + x^2 ) \frac{du}{dx} \right) = f(x) ,~~~ 
0 \leq x \leq 1 ,
\]
\[
u(0) = 1,~~~u'(1) = 0 .
\]
\begin{description}
\item[(a)] Write down a set of difference equations for this problem.
Be sure to show how you do the differencing at the endpoints.
[Note:  It is better {\bf not} to rewrite 
$\frac{d}{dx} ( ( 1+ x^2 ) \frac{du}{dx} )$ as $(1 + x^2 ) u'' (x) + 2x u'(x)$;
leave the equation in the form above.]
\item[(b)] Write a MATLAB code to solve the difference equations.
You can test your code on a problem where you know the solution
by choosing a function $u(x)$ that satisfies the boundary conditions
and determining what $f(x)$ must be in order for $u(x)$ to solve the
problem.  Try $u(x) = (1-x )^2$.  Then $f(x) = 2( 3 x^2 - 2 x + 1 )$.
\item[(c)] Try several different values for the mesh size $h$.  Based
on your results, what would you say is the order of accuracy of your
method?
\end{description}


{\bf Solution:}
\begin{enumerate}
\item
For the interior points:
\[\frac{(1+x_{j+1/2}^2)\frac{du}{dx}_{|j+1/2} -(1+x_{j-1/2}^2)\frac{du}{dx}_{|j-1/2} }{h}= f(x_{j})\]
\[\frac{(1+x_{j+1/2}^2)(u_{j+1}-u_{j}) -(1+x_{j-1/2}^2)(u_{j}-u_{j-1})  }{h^2}= f(x_{j})\]

For the end points:
\[u_0=1\]
And
\[u'(1)=0\]
So, using a second order backward scheme
\[3u_{end}-4u_{end-1}+u_{end-2}=0\]
Since all the differencing is second order, the local error is expected to be O($h^2$). The global error  would be slightly bigger.


\item

	\begin{lstlisting}[style=MyPythonstyle]
def rho(x):
    return 1+x**2
    
def tridiag(a, b, c, k1=-1, k2=0, k3=1):
    return np.diag(a, k1) + np.diag(b, k2) + np.diag(c, k3)

def u(x):
    return (1-x)**2
U=[]
def f(x):
    return 6*x**2 - 4*x + 2
wer=[]
error=[]
for N in [10,100,1000,10000]:
    a=[]
    b=[]
    c=[]
    B=[]
    h=1/N
    U=[]
    
    for i in range(N-1):
        a.append(rho((i+0.5)*h)/(h**2))
        b.append((-rho((i+0.5)*h)-rho((i+1.5)*h))/(h**2))
        c.append(rho((i+1.5)*h)/(h**2))
        B.append(f((i+1)*h))
        U.append(u((i+1)*h))
    #b=-a-c
    a=a+[1]
    c=[0]+c
    b=[1]+b+[-1]
    B=[1]+B+[0]
    U=[1]+U+[0]
    A = tridiag(a, b, c)
    
    A[-1,-1]=3
    A[-1,-2]=-4
    A[-1,-3]=1
    
    Y= np.linalg.solve(A, B)
    eee=U-Y
    wer.append([h,np.linalg.norm(eee)])
    error+=[np.linalg.norm(eee)]
print(tabulate(wer,headers=['h', 'Error'],tablefmt="latex"))
\end{lstlisting}

\item

\begin{tabular}{rr}
\hline
      h &       Error \\
\hline
 0.1    & 0.00555442  \\
 0.01   & 0.000167169 \\
 0.001  & 5.26522e-06 \\
 0.0001 & 1.58201e-07 \\
\hline
\end{tabular}

Dividing the errors givs us a jump of 33 $\approx 10^{1.5}$ showing that the error is O($h^{1.5}$) in between
 O(h) and O($h^2$).

\end{enumerate}
\end{enumerate}

\end{document}
