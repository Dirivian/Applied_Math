%  This is a LaTex file.

%  Homework for the course "AMath 585:  Applied Linear Algebra and Numerical Analysis", 
%  Autumn quarter, 2009, Anne Greenbaum.


%   A latex format for making homework assignments.


\documentclass[letterpaper,12pt]{article}

%          The page format, somewhat wider and taller page than in art12.sty.

\topmargin -0.1in \headsep 0in \textheight 8.9in \footskip 0.6in
\oddsidemargin 0in  \evensidemargin 0in  \textwidth 6.5in
\usepackage{graphicx}
\usepackage{listings}
\usepackage{caption}
\usepackage{subcaption}
\usepackage{color}
\usepackage{float}
\definecolor{keywords}{RGB}{255,0,90}
\definecolor{comments}{RGB}{0,0,113}
\definecolor{red}{RGB}{160,0,0}
\definecolor{green}{RGB}{0,150,0}
\definecolor{codegreen}{rgb}{0,0.6,0}
\definecolor{codegray}{rgb}{0.5,0.5,0.5}
\definecolor{codepurple}{rgb}{0.58,0,0.82}
\definecolor{backcolour}{rgb}{0.95,0.95,0.92}
\definecolor{brown}{rgb}{0.59, 0.29, 0.0}
\definecolor{beaublue}{rgb}{0.74, 0.83, 0.9}
\definecolor{orange}{rgb}{1.0, 0.5, 0.0}
\definecolor{darkslategray}{rgb}{0.18, 0.31, 0.31}
\definecolor{deepblue}{rgb}{0,0,0.5}
\definecolor{deepred}{rgb}{0.6,0,0}
\definecolor{deepgreen}{rgb}{0,0.5,0}
\lstdefinestyle{myMatlabstyle}{
	language=Matlab,
	backgroundcolor=\color{white},   
	commentstyle=\color{codegreen},
	keywordstyle=\color{blue},
	identifierstyle=\color{brown},
	numberstyle=\tiny\color{codegray},
	stringstyle=\color{orange},
	basicstyle=\footnotesize,
	breakatwhitespace=false,         
	breaklines=true,                 
	captionpos=b,                    
	keepspaces=true,                 
	numbers=left,                    
	numbersep=5pt,                  
	showspaces=false,                
	showstringspaces=false,
	showtabs=false,                  
	tabsize=2
}
\lstdefinestyle{myPythonstyle}{
	language=Python, 
	basicstyle=\ttfamily\small, 
	keywordstyle=\color{blue},
	commentstyle=\color{green},
	stringstyle=\color{red},
	showstringspaces=false,
	identifierstyle=\color{black},
}
\lstset{language=Matlab,frame=single}
\lstset{language=Python,frame=single}
\usepackage{amsmath}
\usepackage{amsmath,amsfonts,amsthm} % 
\usepackage{epsfig}         % to insert PostScript figures

\begin{document}


%          Definitions of commonly used symbols.



%          The title and header.

\noindent
{\scriptsize AMath 585, Winter 2018} \hfill 

\begin{center}
\large
Assignment 2. 
\normalsize

Jithin D. George, No. 1622555

\end{center}
\noindent
Due Monday, Jan. 22.
\vspace{.3in}


%           The questions!



\noindent
Reading:  Through Sec. 2.15.  (Page numbers and equation numbers in this 
assignment refer to the text.)
\vspace{.5cm}

\begin{enumerate}
\item (Inverse matrix and Green's functions)
\begin{enumerate}
\item Write out the $5\times 5$
matrix $A$ from (2.43) for the boundary value problem
$u''(x)=f(x)$ with $u(0)=u(1)=0$ for  $h = 0.25$.


{\bf Solution:}


\[A = \begin{bmatrix}
        1. & 0. & 0. & 0. & 0.\\
        0.25 & -0.5 & 0.25 & 0. & 0.\\ 
        0. & 0.25 & -0.5 & 0.25 & 0.\\ 
        0. & 0. & 0.25 & -0.5 & 0.25\\ 
        0. & 0. & 0. & 0. & 1.\\
        \end{bmatrix} \]


\item Write out the $5\times 5$
inverse matrix $A^{-1}$ explicitly for this problem.

{\bf Solution:}

\[\begin{bmatrix}
 1. & 0. & 0. & 0. & 0.\\  
 0.75 & -3. & -2. & -1. & 0.25\\
 0.5 & -2. & -4. & -2. & 0.5\\
 0.25 & -1. & -2. & -3. & 0.75\\
 0. & 0. & 0. & 0. & 1.\\
 \end{bmatrix}\]

\item
If $f(x)=x$, determine the discrete approximation to the solution of the
boundary value problem on this grid and sketch this solution and the five
Green's functions whose sum gives this solution.
\end{enumerate}

\item (Another way of analyzing the error using Green's functions)
The {\em composite trapezoid rule} for integration approximates the
integral from $a$ to $b$ of a function $g$ by dividing the interval
into segments of length $h$ and approximating the integral over each
segment by the integral of the linear function that matches $g$ at
the endpoints of the segment. (For $g > 0$, this is the area of
the trapezoid with height $g( x_j )$ at the left endpoint $x_j$
and height $g( x_{j+1} )$ at the right endpoint $x_{j+1}$.)  Letting
$h = (b-a)/(m+1)$ and $x_j = a + jh$, $j = 0,1, \ldots , m, m+1$:
\[
\int_a^b g(x)\,dx \approx  h \sum_{j=0}^m \frac{g( x_j )+g( x_{j+1} )}{2} 
     =  h \left[ \frac{g( x_0 )}{2} + \sum_{j=1}^m g( x_j ) + 
                   \frac{g( x_{m+1} )}{2} \right] .
\]

\begin{enumerate}
\item
Assuming that $g$ is sufficiently smooth, show that the error in the 
composite trapezoid rule approximation to the integral is $O( h^2 )$.
[Hint:  Show that the error on each subinterval is $O( h^3 )$.]

\item
Recall that the true solution of the boundary value problem $u'' (x) = f(x)$,
$u(0) = u(1) = 0$ can be written as
\begin{equation}
u(x) = \int_0^1 f( \bar{x} ) G(x; \bar{x} )\,d \bar{x} , \label{1}
\end{equation}
where $G(x; \bar{x})$ is the Green's function corresponding to
$\bar{x}$.  The finite difference approximation $u_i$ to $u( x_i )$, 
using the centered finite difference scheme in (2.43), is
\begin{equation}
u_i = h \sum_{j=1}^m f( x_j ) G( x_i ; x_j ) ,~~i=1, \ldots , m . \label{2}
\end{equation}
Show that formula (\ref{2}) is the trapezoid rule approximation to 
the integral in (\ref{1}) when $x = x_i$, and conclude from this that the 
error in the finite difference approximation is $O( h^2 )$ at each node $x_i$. 
[Recall:  The Green's function $G( x ; x_j )$ has a {\em discontinuous}
derivative at $x = x_j$.  Why does this not degrade the accuracy of the
composite trapezoid rule?]
\end{enumerate}

\item (Green's function with Neumann boundary conditions)
\begin{enumerate}
\item
Determine the Green's functions for the two-point boundary
value problem $u''(x) = f(x)$ on $0<x<1$ with a Neumann boundary condition
at $x=0$ and a Dirichlet condition at $x=1$, i.e, find the function
$G(x,\bar x)$ solving
\[
u''(x) = \delta(x-\bar x), \quad u'(0)=0, \quad u(1)=0
\]
and the functions $G_0(x)$ solving
\[
u''(x) = 0, \quad u'(0)=1, \quad u(1)=0
\]
and $G_1(x)$ solving
\[
u''(x) = 0, \quad u'(0)=0, \quad u(1)=1.
\]


{\bf Solution:}

Solving the odes, we have
\[G_0(x) = x-1\]
\[G_1(x) = 1\]
\item
Using this as guidance, find the general formulas for the elements of the
inverse of the matrix in equation (2.54).  Write out the $5\times 5$ matrices
$A$ and $A^{-1}$ for the case $h=0.25$.

{\bf Solution:}


\[A = \begin{bmatrix}
        3. & -4. & 1. & 0. & 0.\\
        0.25 & -0.5 & 0.25 & 0. & 0.\\ 
        0. & 0.25 & -0.5 & 0.25 & 0.\\ 
        0. & 0. & 0.25 & -0.5 & 0.25\\ 
        0. & 0. & 0. & 0. & 1.\\
        \end{bmatrix} \]
        
\[A^{-1} =\begin{bmatrix}  2. & -20. & -8. & -4. & 1.\\  1.5 & -18. & -8. & -4. & 1.\\  1. & -12. & -8. & -4. & 1.\\  0.5 & -6. & -4. & -4. & 1.\\  0. & 0. & 0. & 0. & 1.\\

\end{bmatrix}\]

\end{enumerate}


\item (Solvability condition for Neumann problem)
Determine the null space of the matrix $A^T$, where $A$ is given in
equation (2.58), and verify that the condition (2.62) must hold for the
linear system to have solutions.

\item (Symmetric tridiagonal matrices)
\begin{enumerate}
\item
Consider the {\bf Second approach} described on p.~31 for dealing with
a Neumann boundary condition.  If we use this technique to approximate
the solution to the boundary value problem $u'' (x) = f(x)$, 
$0 \leq x \leq 1$, $u' (0) = \sigma$, $u(1) = \beta$, then the resulting
linear system $A {\bf u} = {\bf f}$ has the following form:
\[
\frac{1}{h^2} \left( \begin{array}{ccccc}
-h & h  &        &        &      \\
1  & -2 & 1      &        &      \\
   & 1  & \ddots & \ddots &      \\
   &    & \ddots & \ddots & 1    \\
   &    &        & 1      & -2 \end{array} \right)
\left( \begin{array}{c} u_0 \\ u_1 \\ \vdots \\ u_{m-1} \\ u_{m} \end{array} 
\right) =
\left( \begin{array}{c} \sigma + (h/2) f( x_0 ) \\ f( x_1 ) \\ \vdots \\ 
f( x_{m-1} ) \\ f( x_m ) - \beta / h^2 \end{array} \right) .
\]
Show that the above matrix is similar to a symmetric tridiagonal matrix 
via a {\em diagonal} similarity transformation; that is, there is a diagonal
matrix $D$ such that $D A D^{-1}$ is symmetric.
\item
Consider the {\bf Third approach} described on pp.~31-32 for dealing with
a Neumann boundary condition.  [{\bf Note:} If you have an older edition of the text,
there is a typo in the matrix (2.57) on p.~32.  There should be a row above what is written
there that has entries $\frac{3}{2} h$, $-2h$, and $\frac{1}{2} h$
in columns $1$ through $3$ and $0$'s elsewhere.  I believe this was corrected in newer editions.]  
Show that if we use that first equation
(given at the bottom of p. 31) to eliminate $u_0$ and we also eliminate
$u_{m+1}$ from the equations by setting it equal to $\beta$ and modifying
the right-hand side vector accordingly, then we obtain an $m$ by $m$
linear system $A {\bf u} = {\bf f}$, where $A$ is similar to a symmetric
tridiagonal matrix via a diagonal similarity transformation.
\end{enumerate}

\end{enumerate}

\end{document}
