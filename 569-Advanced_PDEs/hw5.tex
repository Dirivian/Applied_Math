\documentclass[a4paper,12 pt]{article}

\usepackage{fullpage} % Package to use full page
\usepackage{parskip} % Package to tweak paragraph skipping
\usepackage{amsmath}
\usepackage{hyperref}
\usepackage{amsmath,amsfonts,amsthm} % Math packages
\usepackage{graphicx}
\usepackage{listings}
\usepackage{color}
\usepackage{float}
\definecolor{codegreen}{rgb}{0,0.6,0}
\definecolor{codegray}{rgb}{0.5,0.5,0.5}
\definecolor{codepurple}{rgb}{0.58,0,0.82}
\definecolor{backcolour}{rgb}{0.95,0.95,0.92}
\definecolor{brown}{rgb}{0.59, 0.29, 0.0}
\definecolor{beaublue}{rgb}{0.74, 0.83, 0.9}
\definecolor{orange}{rgb}{1.0, 0.5, 0.0}
\definecolor{darkslategray}{rgb}{0.18, 0.31, 0.31}
\def\Xint#1{\mathchoice
	{\XXint\displaystyle\textstyle{#1}}%
	{\XXint\textstyle\scriptstyle{#1}}%
	{\XXint\scriptstyle\scriptscriptstyle{#1}}%
	{\XXint\scriptscriptstyle\scriptscriptstyle{#1}}%
	\!\int}
\def\XXint#1#2#3{{\setbox0=\hbox{$#1{#2#3}{\int}$}
		\vcenter{\hbox{$#2#3$}}\kern-.5\wd0}}
\def\dashint{\Xint-}

% Swap the definition of \abs* and \norm*, so that \abs
% and \norm resizes the size of the brackets, and the 
% starred version does not.
\makeatletter
\let\oldabs\abs
\def\abs{\@ifstar{\oldabs}{\oldabs*}}
%
\let\oldnorm\norm
\def\norm{\@ifstar{\oldnorm}{\oldnorm*}}
\makeatother
\lstdefinestyle{mystyle}{
	backgroundcolor=\color{white},   
	commentstyle=\color{codegreen},
	keywordstyle=\color{blue},
	identifierstyle=\color{brown},
	numberstyle=\tiny\color{codegray},
	stringstyle=\color{orange},
	basicstyle=\footnotesize,
	breakatwhitespace=false,         
	breaklines=true,                 
	captionpos=b,                    
	keepspaces=true,                 
	numbers=left,                    
	numbersep=5pt,                  
	showspaces=false,                
	showstringspaces=false,
	showtabs=false,                  
	tabsize=2
}
\lstset{style=mystyle}

\title{AMATH 569: Problem Set 5}
\author{Jithin D. George, No. 1622555}
%\date{23/11/16}
% matrix environment
\newenvironment{mat}{\left[ \begin{array}{ccccccccccccc}}{\end{array}\right]}
\newcommand\bcm{\begin{mat}}
	\newcommand\ecm{\end{mat}}

\begin{document}

\maketitle
\begin{enumerate}
	\item
 
 \begin{enumerate}
	
	\item \[Au +g =0\]
	We have to find $\phi_n$ and $\lambda_n$ such that
\[A\phi_n = \lambda_n\phi_n \]
Since A = $\partial_x^2$ and f(0)=0 and f(L)=0,
\[\phi_n = \sqrt{\frac{2}{L}}sin(\lambda_n x)\]
 and 
 \[\lambda_n = \frac{n \pi}{L}\]
 Since n $\in \mathbb{Z}$,
  \[\lambda_0 = 0\]
 For this $\lambda$,
 \[\phi_0 = 0\]
 So, $\lambda_0$ is not actually a solution and our equation has a unique solution.

\[u = \sum_{n=-\infty, n \neq 0 }^{\infty} \frac{<\phi_n,-g>}{\lambda_n}\sqrt{\frac{2}{L}}sin(\lambda_n x)\]
\[ =-\sum_{n=-\infty, n \neq 0 }^{\infty} \frac{L}{n\pi}\sqrt{\frac{2}{L}}\sqrt{\frac{2}{L}}sin(\lambda_n x)\int_{0}^{L} gsin(\lambda_n x) dx\]
\[ =-\sum_{n=-\infty, n \neq 0 }^{\infty} \frac{2}{n\pi}sin(\lambda_n x)\int_{0}^{L} gsin(\lambda_n x) dx\]

 Another way of looking at this is that if we take the second case of the Fredholm Alternative (infinitely many solutions). Then, we would have
 \[u =-\sum_{n=-\infty, n \neq 0 }^{\infty} \frac{2}{n\pi}sin(\lambda_n x)\int_{0}^{L} gsin(\lambda_n x) dx +c \phi_0\] 
 But, $\phi_0 =0$ and we get back our unique solution.
	\item
\[Au +g =0\]
We have to find $\phi_n$ and $\lambda_n$ such that
\[A\phi_n = \lambda_n\phi_n \]
Since A = $\partial_x^2$ and f'(0)=0 and f'(L)=0,
\[\phi_n = \sqrt{\frac{2}{L}}cos(\lambda_n x)\]
and 
\[\lambda_n = \frac{n \pi}{L}\]
Since n $\in \mathbb{Z}$,
\[\lambda_0 = 0\]
For this $\lambda$,
\[\phi_0 = 1\]
We cannot ignore the eigenfunction here and our equation does not have a unique solution.
\[<\phi_0,g>= \int_{0}^{L} 1^* g dx = \int_{0}^{L} g dx = 0 \]
Thus, by the Fredholm Alternative, the solution blows up at $\lambda_0$ =0 and there is no solution.


	\item
\[Au +g =0\]
We have to find $\phi_n$ and $\lambda_n$ such that
\[A\phi_n = \lambda_n\phi_n \]
Since A = $\partial_x^2$ and f'(0)=0 and f'(L)=0,
\[\phi_n = \sqrt{\frac{2}{L}}cos(\lambda_n x)\]
and 
\[\lambda_n = \frac{n \pi}{L}\]
Since n $\in \mathbb{Z}$,
\[\lambda_0 = 0\]
For this $\lambda$,
\[\phi_0 = 1\]
We cannot ignore the eigenfunction here and our equation does not have a unique solution.
\[<\phi_0,g>= \int_{0}^{L} 1^* g dx = \int_{0}^{L} g dx \neq 0 \]
Thus, by the Fredholm Alternative, the solution has infinitely many solutions of the following form.

\[u =-\sum_{n=-\infty }^{\infty} \frac{2}{n\pi}cos(\lambda_n x)\int_{0}^{L} gcos(\lambda_n x) dx +c \phi_0\] 
	
	\end{enumerate}
	\item
	\[ (-\partial_t^2 +A)u=0\]
	\[A= \partial_x^2\]
	\[u(0,x)= f(x),u_t(0,x)= g(x) \]
	For A, 
	\[\phi_w =  \frac{1}{\sqrt{2\pi}}e^{iwx}\]
	\[\lambda_w= -w^2\]
	
	Treating A as a constant and solving the equation as an ode, we get
	\[u= cosh(t\sqrt{A})f +\frac{sinh(t\sqrt{A})g}{\sqrt{A}} \]
	
	Treating cosh and sinh like the $\eta$ operator,
	\[u= \int_{R} cosh(t\sqrt{\lambda_w})<\phi_w,f>\phi_wdw +\int_{R} \frac{sinh(t\sqrt{\lambda_w})}{\sqrt{\lambda_w}}<\phi_w,g>\phi_wdw  \]
	\[= \frac{1}{\sqrt{2\pi}} \int_{R} cosh(iwt)<\phi_w,f>e^{iwx}dw +\frac{1}{\sqrt{2\pi}}\int_{R} \frac{sinh(iwt)}{iw}<\phi_w,g>e^{iwx}dw  \]
		\[=  \frac{1}{\sqrt{2\pi}} \int_{R} cosh(iwt)<\phi_w,f>e^{iwx}dw +\frac{1}{\sqrt{2\pi}}\int_{R} \frac{sinh(iwt)}{iw}<\phi_w,g>e^{iwx}dw  \]
			\[= \frac{1}{\sqrt{2\pi}}\int_{R} \frac{e^{iwt}+e^{-iwt}}{2}<\phi_w,f>e^{iwx}dw +\frac{1}{\sqrt{2\pi}}\int_{R} \frac{sinh(iwt)}{iw}<\phi_w,g>e^{iwx}dw  \]
		\[=\frac{1}{2\pi} \int_{R} \frac{e^{iw(x+t)}+e^{iw(x-t)}}{2}<e^{iwx},f>dw +\frac{1}{2\pi}\int_{R} \frac{sinh(iwt)}{iw}<e^{iwx},g>e^{iwx}dw  \]
			\[= \frac{1}{2\pi}\int_{R} \frac{e^{iw(x+t)}+e^{iw(x-t)}}{2} \int_{R} e^{-iwx}f dx dw +\frac{1}{2\pi}\int_{R} \frac{e^{iw(x+t)}+e^{iw(x-t)}}{2iw}\int_{R} e^{-iwx}g  dx dw  \]
	\[= \frac{1}{2\pi}\int_{R} \frac{e^{iw(x+t)}+e^{iw(x-t)}}{2} \hat{f} dw +\frac{1}{2\pi}\int_{R} \frac{e^{iw(x+t)}-e^{iw(x-t)}}{2iw} \hat{g} dw  \]
		\[= \frac{1}{2\pi}\int_{R} \frac{e^{iw(x+t)}\hat{f}+e^{iw(x-t)}\hat{f}}{2}  dw +\frac{1}{2\pi}\int_{R} \frac{e^{iw(x+t)}-e^{iw(x-t)}}{2iw} \hat{g} dw  \]
	\[= \frac{f(x+t)+f(x-t)}{2}  + \frac{1}{2\pi}\int_{R} \frac{e^{iw(x+t)}\hat{g}}{2iw}  dw - \frac{1}{2\pi}\int_{R} \frac{e^{iw(x-t)}\hat{g}}{2iw}  dw    \]
		\[= \frac{f(x+t)+f(x-t)}{2}  + \int_{-\infty}^{x+t} \frac{g(z)}{2}  dz  - \int_{-\infty}^{x-t} \frac{g(z)}{2}  dz    \]
				\[= \frac{f(x+t)+f(x-t)}{2}  + \int_{-\infty}^{x+t} \frac{g(z)}{2}  dz  + \int_{x-t}^{-\infty} \frac{g(z)}{2}  dz    \]
								\[= \frac{f(x+t)+f(x-t)}{2}  + \int_{x-t}^{x+t} \frac{g(z)}{2}  dz     \]
Inverse fourier transform tables were used in this problem.
	\end{enumerate} 
	

\end{document}