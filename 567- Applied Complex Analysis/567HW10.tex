\documentclass[a4paper]{article}

\usepackage{fullpage} % Package to use full page
\usepackage{parskip} % Package to tweak paragraph skipping
\usepackage{amsmath}
\usepackage{hyperref}
\usepackage{amsmath,amsfonts,amsthm} % Math packages
\usepackage{graphicx}
\usepackage{listings}
\usepackage{color}
\usepackage{float}
\definecolor{codegreen}{rgb}{0,0.6,0}
\definecolor{codegray}{rgb}{0.5,0.5,0.5}
\definecolor{codepurple}{rgb}{0.58,0,0.82}
\definecolor{backcolour}{rgb}{0.95,0.95,0.92}
\definecolor{brown}{rgb}{0.59, 0.29, 0.0}
\definecolor{beaublue}{rgb}{0.74, 0.83, 0.9}
\definecolor{orange}{rgb}{1.0, 0.5, 0.0}
\definecolor{darkslategray}{rgb}{0.18, 0.31, 0.31}
\def\Xint#1{\mathchoice
	{\XXint\displaystyle\textstyle{#1}}%
	{\XXint\textstyle\scriptstyle{#1}}%
	{\XXint\scriptstyle\scriptscriptstyle{#1}}%
	{\XXint\scriptscriptstyle\scriptscriptstyle{#1}}%
	\!\int}
\def\XXint#1#2#3{{\setbox0=\hbox{$#1{#2#3}{\int}$}
		\vcenter{\hbox{$#2#3$}}\kern-.5\wd0}}
\def\dashint{\Xint-}

% Swap the definition of \abs* and \norm*, so that \abs
% and \norm resizes the size of the brackets, and the 
% starred version does not.
\makeatletter
\let\oldabs\abs
\def\abs{\@ifstar{\oldabs}{\oldabs*}}
%
\let\oldnorm\norm
\def\norm{\@ifstar{\oldnorm}{\oldnorm*}}
\makeatother
\lstdefinestyle{mystyle}{
	backgroundcolor=\color{white},   
	commentstyle=\color{codegreen},
	keywordstyle=\color{blue},
	identifierstyle=\color{brown},
	numberstyle=\tiny\color{codegray},
	stringstyle=\color{orange},
	basicstyle=\footnotesize,
	breakatwhitespace=false,         
	breaklines=true,                 
	captionpos=b,                    
	keepspaces=true,                 
	numbers=left,                    
	numbersep=5pt,                  
	showspaces=false,                
	showstringspaces=false,
	showtabs=false,                  
	tabsize=2
}
\lstset{style=mystyle}

\title{AMATH 567: Problem Set 10}
\author{Jithin D. George, No. 1622555}
%\date{23/11/16}

\begin{document}

\maketitle
\begin{enumerate}

	
	\item 
\begin{enumerate}
	
	\item
		\[\psi^{+}(x)= \lim_{\epsilon \to 0+} \frac{1}{2\pi i} \int_{-\infty}^\infty \frac{f(t)}{t - (x+i\epsilon)}dt\]
		The function inside the integral has a singularity
		\[P^{+}= \lim_{\epsilon \to 0+} \frac{1}{2\pi i} \int_{-\infty}^\infty \frac{1}{t - (x+i\epsilon)}dt\]	
		\[P^{+}\psi^{+}(x)= \lim_{\epsilon \to 0+} \frac{1}{2\pi i} \int_{-\infty}^\infty \frac{\psi^{+}(t)}{t - (x+i\epsilon)}dt\]	
We decide to do the integral in the complex plan with the upper semicircle as the contour because $\psi^{+}(z) $ is analytic in the upper half plane. Assuming the function on the arc decays as the arc goes to $\infty$ , we can ignore the integral along it.	
		\[P^{+}\psi^{+}(x)= \lim_{\epsilon \to 0+} \frac{1}{2\pi i} Res_{t = z+i\epsilon} \frac{\psi^{+}(t)}{t - (x+i\epsilon)}dt\]		
		\[P^{+}\psi^{+}(x)= \lim_{\epsilon \to 0+} \psi^{+}(x+i\epsilon) =\psi^{+}(x) \]	
		
			\[\psi^{-}(x)= \lim_{\epsilon \to 0+} \frac{1}{2\pi i} \int_{-\infty}^\infty \frac{f(t)}{t - (x-i\epsilon)}dt\]		
	\[P^{-}= \lim_{\epsilon \to 0+} \frac{1}{2\pi i} \int_{-\infty}^\infty \frac{1}{t - (x-i\epsilon)}dt\]	
	\[P^{-}\psi^{-}(x)= \lim_{\epsilon \to 0+} \frac{1}{2\pi i} \int_{-\infty}^\infty \frac{\psi^{-}(t)}{t - (x-i\epsilon)}dt\]	
We decide to do the integral in the complex plan with the lower semicircle as the contour because $\psi^{-}(z) $ is analytic in the lower half plane. Assuming the similar decay as before,
		\[P^{-}\psi^{-}(x)= -\lim_{\epsilon \to 0+} \frac{1}{2\pi i} Res_{t = x-i\epsilon} \frac{\psi^{-}(t)}{t - (x-i\epsilon)}dt\]
		The minus sign comes because the integral in the lower semicircle is clockwise.		
		\[P^{-}\psi^{-}(x)= -\lim_{\epsilon \to 0+} \psi^{-}(x-i\epsilon) =-\psi^{-}(x) \]
Also, 		
	\[P^{-}\psi^{+}(x)= \lim_{\epsilon \to 0+} \frac{1}{2\pi i} \int_{-\infty}^\infty \frac{\psi^{+}(t)}{t - (x-i\epsilon)}dt =0\]
	since there is no residue in the upper half plane.
	
Similarly, 	
	\[P^{+}\psi^{-}(x) =0\]
	
	\item  \[\psi^{+}(x) -\psi^{-}(x)= \frac{1}{x^4+1}\]
	\[P^{+}\psi^{+}(x) -P^{+}\psi^{-}(x)= P^{+}\frac{1}{x^4+1}\]
	\[P^{+}\psi^{+}(x)=P^{+}(\frac{1}{x^4+1})\]
		\[\psi^{+}(x)=P^{+}(\frac{1}{x^4+1})\]		
 	\[\psi^{-}(x)=P^{-}(\frac{1}{x^4+1})\]	
 		\[\psi^{+}(x)=\lim_{\epsilon \to 0+}\frac{1}{2\pi i}\int_{-\infty}^\infty \frac{1}{(t^4+1)(t - (x+i\epsilon))}dt\]
	
  		Taking the upper semicircle as the contour,
    		\[\psi^{+}(x)=\lim_{\epsilon \to 0+} (\lim_{t \to \frac{1+i}{\sqrt{2}}} \frac{1}{(t+\frac{1+i}{\sqrt{2}})(t-\frac{1-i}{\sqrt{2}})(t-\frac{-1+i}{\sqrt{2}})(t - (x+i\epsilon))})\]	
   	\[+\lim_{\epsilon \to 0+} (\lim_{t \to \frac{-1+i}{\sqrt{2}}} \frac{1}{(t+\frac{1+i}{\sqrt{2}})(t-\frac{1-i}{\sqrt{2}})(t-\frac{1+i}{\sqrt{2}})(t - (x+i\epsilon))})\]
   	\[+\lim_{\epsilon \to 0+} (\lim_{t \to x+i\epsilon} \frac{1}{(t+\frac{1+i}{\sqrt{2}})(t-\frac{1-i}{\sqrt{2}})(t-\frac{-1+i}{\sqrt{2}})(t-\frac{1-i}{\sqrt{2}})})\]    									
   	
   	\[\psi^{+}(x)= \frac{1}{(2(1+i)i(1+i-\sqrt{2}x))}-\frac{1}{(2(-1+i)i(-1+i-\sqrt{2}x))} +\frac{1}{x^4+1}\]
   	
 		\[\psi^{-}(x)=\lim_{\epsilon \to 0+}\frac{1}{2\pi i}\int_{-\infty}^\infty \frac{1}{(t^4+1)(t - (x-i\epsilon))}dt\]
 		
 		Taking the lower semicircle as the contour,
 		\[\psi^{-}(x)=\lim_{\epsilon \to 0+} (\lim_{t \to -\frac{1+i}{\sqrt{2}}} \frac{1}{(t-\frac{1+i}{\sqrt{2}})(t-\frac{1-i}{\sqrt{2}})(t-\frac{-1+i}{\sqrt{2}})(t - (x+i\epsilon))})\]	
 		\[+\lim_{\epsilon \to 0+} (\lim_{t \to \frac{1-i}{\sqrt{2}}} \frac{1}{(t+\frac{1+i}{\sqrt{2}})(t-\frac{-1+i}{\sqrt{2}})(t-\frac{1+i}{\sqrt{2}})(t - (x+i\epsilon))})\]
 		\[+\lim_{\epsilon \to 0+} (\lim_{t \to x-i\epsilon} \frac{1}{(t+\frac{1+i}{\sqrt{2}})(t-\frac{1-i}{\sqrt{2}})(t-\frac{-1+i}{\sqrt{2}})(t-\frac{1-i}{\sqrt{2}})})\]    									
 		
 		\[\psi^{-}(x)= \frac{1}{(2(1+i)i(1+i+\sqrt{2}x))}-\frac{1}{(2(1-i)i(1-i-\sqrt{2}x))} +\frac{1}{x^4+1}\]  
 		
 		\[\psi^{+}(x)-\psi^{-}(x)=	\frac{1}{(2(1+i)i(1+i-\sqrt{2}x))}-\frac{1}{(2(-1+i)i(-1+i-\sqrt{2}x))} \]
 		\[- \frac{1}{(2(1+i)i(1+i+\sqrt{2}x))}+\frac{1}{(2(1-i)i(1-i-\sqrt{2}x))} \]
 
From Mathematica, we get that the right hand side is the partial fraction expansion of $\frac{1}{x^4+1}$. So,	
 \[\psi^{+}(x)-\psi^{-}(x)= \frac{1}{x^4+1}\] 	
\item
		\[\psi^{+}(x)= \lim_{\epsilon \to 0+} \frac{1}{2\pi i} \int_{-\infty}^\infty \frac{f(t)}{t - (x+i\epsilon)}dt \] 
		\vspace{30mm}
		
		$C_R$ is a tiny circle centered at x with radius $\delta$.There is no singularity in the region A because the singularity is currently at x+i$\epsilon$. So, the integral along the dotted line is the same as the integral along $C_R$.
		\[\psi^{+}(x)=	\lim_{\epsilon \to 0+} \frac{1}{2\pi i}\lim_{\delta\to 0} \big(\int_{-\infty}^{x-\delta} \frac{f(t)}{t - (x+i\epsilon)}dt +\int_{x+\delta}^\infty \frac{f(t)}{t - (x+i\epsilon)}dt + \int_{C_R} \frac{f(t)}{t - (x+i\epsilon)}dt \big)\]
				
		\[=	\lim_{\epsilon \to 0+} \frac{1}{2\pi i} \dashint_{-\infty}^{\infty} \frac{f(t)}{t - (x+i\epsilon)}dt  +  \lim_{\epsilon \to 0+} \frac{1}{2\pi i}\lim_{\delta\to 0} \int_{C_R} \frac{f(t)}{t - (x+i\epsilon)}dt\]	
Switching the limits and using the baby limit theorem on the second term,				
\[=	\lim_{\epsilon \to 0+} \frac{1}{2\pi i} \dashint_{-\infty}^{\infty} \frac{f(t)}{t - (x+i\epsilon)}dt  + \frac{1}{2\pi i}\pi i f(x) \]
\[=	 \frac{1}{2\pi i} \dashint_{-\infty}^{\infty} \frac{f(t)}{t - x}dt  + \frac{1}{2} f(x) \]	

Similarly, for $\psi^{-}$, the contour $C_R$  is in the upper plane making the integral along it negative.
\[\psi^{-}=	 \frac{1}{2\pi i} \dashint_{-\infty}^{\infty} \frac{f(t)}{t - x}dt - \frac{1}{2} f(x) \]	

The first term is closely related to the Hilbert Transform
\[H(f(x)) = \frac{1}{\pi } \dashint_{-\infty}^{\infty} \frac{f(t)}{t - x}dt\] 
\[\psi^{+}=	 \frac{1}{2i}  H(f(x)) + \frac{1}{2} f(x) \]	
\[\psi^{-}=	 \frac{1}{2i}  H(f(x)) - \frac{1}{2} f(x) \]		 
\end{enumerate}
\item		
\begin{enumerate}
	
	\item
	\[\hat{F}(k)=\int_{-\infty}^\infty e^{-ikx}f(x)dx\]
	\[=\int_{-\infty}^0 e^{-ikx}0dx+\int_{0+}^\infty e^{-ikx}e^{-x}dx\]
		\[=0+\int_{0+}^\infty e^{(-ik-1)x}dx\]
	\[=\frac{1}{1+ik}\]
		\item
			\[\hat{F}(k)=\frac{1}{1+ik}\]
			
			This has a singularity at  k=i.
\[	f(x) = \frac{1}{2\pi}\int_{-\infty}^\infty\frac{e^{ikx}}{1+ik}dk \]
Since this is tough to integrate on the real line, we use integrals along complex contours. 
The numerator is
\[ e^{ikx}= e^{i(k_x+ik_y)x} =e^{ik_xx}e^{-k_yx}\]
When x $>$ 0, this decays to 0 for positive $k_y$
So, we take the closed semicircle in the upper k plane as our contour. Since the integral over the arc decays to 0 for large k and there is only residue at i,
\[			 \int_{-\infty}^\infty \frac{e^{ikx}}{1+ik}dk= Res_{k=i}\frac{e^{ikx}}{1+ik} \]
\[			= 2 \pi i \lim_{k \to i} (k-i)\frac{e^{ikx}}{1+ik} \]
\[			= 2 \pi i  \frac{e^{-x}}{i} \]
\[			= 2 \pi e^{-x} \]

When x $<$ 0, we have to use negative $k_y$ for the decay,
So, we take the closed semicircle in the lower k plane as our contour. Since there is no residue in it,
\[			 \int_{-\infty}^\infty \frac{e^{ikx}}{1+ik}dk= 0 \]

There is obviously a jump at x=0.
\[	f(0) = \frac{1}{2\pi} \dashint_{-\infty}^\infty\frac{1}{1+ik}dk =\frac{1}{2\pi} \dashint_{-\infty}^\infty\frac{1-ik}{1+k^2}dk =\frac{1}{2\pi} \dashint_{-\infty}^\infty\frac{1}{1+k^2}dk  -\frac{i}{2\pi} \dashint_{-\infty}^\infty\frac{k}{1+k^2}dk \]
\[=\frac{1}{2\pi} \dashint_{-\infty}^\infty\frac{1}{1+k^2}dk  -\frac{i}{2\pi} \lim_ {R \to \infty}\int_{-R}^R\frac{k}{1+k^2}dk \]
\[=\frac{1}{2\pi} \left[arctan(x)\right]_{-\infty}^{\infty}  -0 \text{(odd function)} \]
\[= \frac{1}{2}\]


\[   
f(x) = 
\begin{cases}
e^{-x} &\quad\text{if x} >0 \\
\frac{1}{2} &\quad\text{if x} =0 \\
0 &\quad\text{if x} <0\\
\end{cases}
\]




\end{enumerate}

\item 
\[f(z)= -4z\]
\[g(z) = e^z-1\] 
On $|z|=1$, \[ \lvert f(z)\rvert =4 \] and \[ \lvert g(z) \rvert \leq \lvert e^{e^{i\theta}}\rvert +1 =e+1 \]
Thus, by Roche's theorem, f(z)+g(z) has same number of zeros inside the circle as f(z) which has exactly one at 0.
 So, $e^z-4z-1$ has exactly one zero inside $|z|=1$
\item \[f(z)= z^4 +z^3 +5z^2+ 2z+4\]
We take the infinite quarter circle.
\vspace{50mm}

On the real line, from 0 to $\infty$, f(z) is real. So , the change in argument is 0.
On the arc, 
\[f(z) = R^4 e^{4i\theta}+R^3 e^{3i\theta}+5R^2 e^{2i\theta}+2R e^{i\theta}+4\] 
\[ = R^4 (e^{4i\theta}+\frac {e^{3i\theta}}{R}+\frac{5e^{2i\theta}}{R^2}+\frac{2 e^{i\theta}}{R^3}+\frac{4}{R^4})\]
As R goes to $\infty$, f(z) has an argument of 4$\theta$.

From 0 to $\frac{\pi}{2}$, the change in argument is $2\pi$.

On the y axis, \[f(z)  = y^4 -iy^3-5y^2+iy+4\]
\[tan(\theta) = \frac{-y^3+y}{y^4-5y^2+4}\]
For very large y, $tan(\theta) =0 $ because of the term on the bottom. This matches with the value of $\theta = 2\pi $ which we got from the arc. Furthermore, the derivative is negative because of the dominance of $y^3$ in the numerator. So, the argument decreases.

At 0, $tan(\theta) =0 $. So, the argument decreases to a value $k\pi$. 

$y^4-5y^2+4$ has zeros at 2,-2,1,-1. As y goes from $\infty$ to 0, $tan(\theta)$ crosses over 2 of these at 2 and 1. Thus,$ tan(\theta) $has two singularities on the y axis.
\vspace{50mm}


Looking at the graph of tan, we see that after crossing two singularities, the argument comes to 0.

Total change in argument is 0. Thus,by the argument principle, f(z) has no roots in the first quadrant.
\item
\begin{enumerate}
	
	\item
\[p(t) =\frac{t f'(t)}{f(t)-w} \]
This has a singularity when f(t) =w. There is only singularity for a particular w because f(t) is bijective.
\[g(w)=\frac{1}{2\pi i}\oint_{C(z_0,R)}\frac{t f'(t)}{f(t)-w}dt\]
because w= f(z), this evaluates to 
\[ =Res_{t=z}\frac{t f'(t)}{f(t)-w}\]
\[ =\lim_{t\to z} (t-z)\frac{t f'(t)}{f(t)-w}\]
\[ =\lim_{t\to z} \frac{t f'(t)}{\frac{f(t)-w}{(t-z)}}\]
\[ = \frac{z f'(z)}{f'(z)}\]
\[= z\]
\item
\[g(w)=\frac{1}{2\pi i}\oint_{C(z_0,R)}\frac{t f'(t)}{f(t)-w}dt\]
\[=\frac{1}{2\pi i}\oint_{C(z_0,R)}\frac{t (t+1)e^t}{te^t-w}dt\]
\[=\frac{1}{2\pi i}\oint_{C(z_0,R)}\frac{ (t+1)}{1-\frac{w}{te^t}}dt\]
\[=\frac{1}{2\pi i}\oint_{C(z_0,R)} (t+1)(1+\frac{w}{te^t}+({\frac{w}{te^t}})^2 +\ldots)dt\]
Except the first term, all term have a singularity at t=0. So, 
\[=\frac{1}{2\pi i} Res_{t \to 0} (t+1)(\frac{w}{te^t}+({\frac{w}{te^t}})^2 +\ldots)\]
\[=w\frac{1}{2\pi i} Res_{t \to 0}(\frac{(t+1)}{te^t})+w^2\frac{1}{2\pi i} Res_{t \to 0}{\frac{(t+1)}{e^{2t}}}+\frac{w^3}{2!}\frac{1}{2\pi i} Res_{t \to 0}{\frac{(t+1)}{e^{3t}}}+\ldots\]
\[ g(w)=\sum_{0}^{\infty} a_n w^n\]
\[ a_0 = 0\]
\[ a_n =
\frac{1}{2\pi i} Res_{t \to 0}\frac{(t+1)}{e^{nt}}\]


\end{enumerate}
	\end{enumerate} 
\end{document}