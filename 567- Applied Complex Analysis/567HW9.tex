\documentclass[a4paper]{article}

\usepackage{fullpage} % Package to use full page
\usepackage{parskip} % Package to tweak paragraph skipping
\usepackage{amsmath}
\usepackage{hyperref}
\usepackage{amsmath,amsfonts,amsthm} % Math packages
\usepackage{graphicx}
\usepackage{listings}
\usepackage{color}
\usepackage{float}
\definecolor{codegreen}{rgb}{0,0.6,0}
\definecolor{codegray}{rgb}{0.5,0.5,0.5}
\definecolor{codepurple}{rgb}{0.58,0,0.82}
\definecolor{backcolour}{rgb}{0.95,0.95,0.92}
\definecolor{brown}{rgb}{0.59, 0.29, 0.0}
\definecolor{beaublue}{rgb}{0.74, 0.83, 0.9}
\definecolor{orange}{rgb}{1.0, 0.5, 0.0}
\definecolor{darkslategray}{rgb}{0.18, 0.31, 0.31}
\def\Xint#1{\mathchoice
	{\XXint\displaystyle\textstyle{#1}}%
	{\XXint\textstyle\scriptstyle{#1}}%
	{\XXint\scriptstyle\scriptscriptstyle{#1}}%
	{\XXint\scriptscriptstyle\scriptscriptstyle{#1}}%
	\!\int}
\def\XXint#1#2#3{{\setbox0=\hbox{$#1{#2#3}{\int}$}
		\vcenter{\hbox{$#2#3$}}\kern-.5\wd0}}
\def\dashint{\Xint-}

% Swap the definition of \abs* and \norm*, so that \abs
% and \norm resizes the size of the brackets, and the 
% starred version does not.
\makeatletter
\let\oldabs\abs
\def\abs{\@ifstar{\oldabs}{\oldabs*}}
%
\let\oldnorm\norm
\def\norm{\@ifstar{\oldnorm}{\oldnorm*}}
\makeatother
\lstdefinestyle{mystyle}{
	backgroundcolor=\color{white},   
	commentstyle=\color{codegreen},
	keywordstyle=\color{blue},
	identifierstyle=\color{brown},
	numberstyle=\tiny\color{codegray},
	stringstyle=\color{orange},
	basicstyle=\footnotesize,
	breakatwhitespace=false,         
	breaklines=true,                 
	captionpos=b,                    
	keepspaces=true,                 
	numbers=left,                    
	numbersep=5pt,                  
	showspaces=false,                
	showstringspaces=false,
	showtabs=false,                  
	tabsize=2
}
\lstset{style=mystyle}

\title{AMATH 567: Problem Set 9}
\author{Jithin D. George, No. 1622555}
%\date{23/11/16}

\begin{document}

\maketitle
\begin{enumerate}
	
	\item
	
From the last exercise, we have, 	
	\[
f(k)= \pi	{\rm Res}_{z=k}f(z)\cot(\pi z),
	\]
	provided $f(z)$ is analytic at $z=k$ ($k \in \mathbb{Z}$).
	
Now that f(z) may have a pole at z=0, this is valid for all $k \neq 0$.

And so,
\[
{\sum_{k=-\infty}^\infty}' \frac{p(k)}{q(k)}=\pi {\sum_{k=-\infty}^\infty}' {\rm
	Res}_{z=k}f(z)\cot(\pi z),
\]

On a square contour, with corners at $(N+1/2)(\pm 1\pm
i)$, we showed that
\[
\lim_{N\rightarrow \infty}
\left|\oint_{\Gamma_N}\frac{p(z)}{q(z)}\cot(\pi z)dz\right|=0.
\]
\[\sum {\rm
	Res} f(z)\cot(\pi z) =0\]
\[\sum_j {\rm
	Res}_{z=z_j}f(z)\cot(\pi z)+{\sum_{k=-\infty}^\infty} {\rm
	Res}_{z=k}f(z)\cot(\pi z)=0\]
As the k=0 term is a pole of f(z), it can be included as one of the $z_j$ to give
\[\sum_j {\rm
	Res}_{z=z_j}f(z)\cot(\pi z)+{\sum_{k=-\infty}^\infty}' {\rm
	Res}_{z=k}f(z)\cot(\pi z)=0\]
\[{\sum_{k=-\infty}^\infty}' {\rm
	Res}_{z=k}f(z)\cot(\pi z)=-\sum_j {\rm
	Res}_{z=z_j}f(z)\cot(\pi z)\]
So, 
\[{\sum_{k=-\infty}^\infty}' \frac{p(k)}{q(k)}=-\pi \sum_j {\rm
		Res}_{z=z_j}f(z)\cot(\pi z)\]
\item 
\begin{enumerate}
	\item
	\[ {\sum_{k=-\infty}^\infty}' \frac{1}{k^2}= -\pi Res_{z=0} \frac{cot(\pi z)}{z^2}\]
	This is a triple pole as cot($\pi z$) is a simple pole at z=0 (last exercise).  
		\[ cot (\pi z) = \frac{1}{\pi z} - \frac{\pi z}{3} - \frac{(\pi z)^3}{45}-\frac{2(\pi z)^5}{945}+\ldots\]
				\[z cot (\pi z) = \frac{1}{\pi } - \frac{\pi z^2}{3} - \frac{\pi^3 z^4}{45}-\frac{2\pi^5 z^6}{945}+\ldots\]
		\[ {\sum_{k=-\infty}^\infty}' \frac{1}{k^2}= -\pi \lim_{z=0}\frac{1}{2!} \frac{d^2 }{dz^2} \left( zcot(\pi z) \right) = \frac{\pi^2}{3}\]
	\item
		\[ {\sum_{k=-\infty}^\infty}' \frac{1}{k^3}= -\pi Res_{z=0} \frac{cot(\pi z)}{z^3}\]
		This is a fourth order pole.
				\[ {\sum_{k=-\infty}^\infty}' \frac{1}{k^3}= -\pi \lim_{z=0} \frac{1}{3!} \frac{d^3 }{dz^3} \left( zcot(\pi z) \right) = 0\]
	\item
			\[ {\sum_{k=-\infty}^\infty}' \frac{1}{k^4}= -\pi Res_{z=0} \frac{cot(\pi z)}{z^4}\]
			This is a fifth order pole.
					\[ {\sum_{k=-\infty}^\infty}' \frac{1}{k^4}= -\pi \lim_{z=0} \frac{1}{4!} \frac{d^4 }{dz^4} \left( zcot(\pi z) \right) = \frac{\pi^4}{45}\]
\end{enumerate}

\item
\[
\frac{\sin \pi z}{\pi z}=1-\frac{\pi^3 z^3}{\pi z 3!} +\ldots=1-\frac{\pi^2 z^2}{ 3!} +\ldots
\]
\[\prod_{j=1}^\infty
\left(1-\frac{z^2}{j^2}\right) = \left(1 -z^2-\frac{z^2}{2^2}-\frac{z^2}{3^2}-\frac{z^2}{4^2} +\ldots \right) = \left(1  -z^2\left(\sum_{k=1}^\infty \frac{1}{k^2}\right) +\ldots \right) \]
\[ \frac{\pi^2}{3!}=\sum_{k=1}^\infty \frac{1}{k^2}\]
\[ \sum_{k=1}^\infty \frac{1}{k^2}=\frac{\pi^2}{6}\]
\item 
\[
I=\int_{-\infty}^\infty \frac{\sin x}{\sinh x}dx.
\]
Sinh(x) has a zero at x =0 so the integral can't be defined. Sin(x) also has a zero there and the function ends up being have a removable singularity at 0. So, we use the principal value integral instead.
\[I_p =\dashint_{-\infty}^\infty  \frac{\sin x}{\sinh x}dx\]
 We use the contour in the figure below.
 
 \vspace{100 mm}
 $C_0$ and $C_{\pi}$ are two semi circles with centers at the singularities 0 and $\pi$ and of radius $\epsilon$  as $ \epsilon \to 0 $ and thus in agreement with our principal value integral definition.
 
 Since the contour encloses no singularity, by Cauchy's theorem,
 \[\left(\int_{-R}^{-\epsilon} +\int_{C_0}+\int_{\epsilon}^{R} +\int_{R}^{R+\pi i}+\int_{R+\pi i}^{\epsilon+\pi i } +\int_{C_{\pi}} +\int_{-\epsilon+\pi i}^{-R+\pi i}+\int_{-R+\pi i}^{R}\right) f(z)dz= 0\]
 
 Taking the curve $C_3$,
 \[z = R+ i\pi t\]
 where t goes from 0 to 1.
 \[\Big\lvert\frac{\sin z}{\sinh z}\Big\rvert = \Big\lvert\frac{(e^{iz}-e^{-iz})}{i(e^{z}-e^{-z})}\Big\rvert\]
 \[  =\Big\lvert\frac{(e^{iR}e^{-\pi t}-e^{-iR}e^{\pi t})}{i(e^{R}e^{i \pi t}-e^{-R}e^{-i \pi t})}\Big\rvert \]
  \[  \leq \Big\lvert\frac{\lvert e^{iR}\rvert\lvert e^{-\pi t}\rvert+\lvert e^{-iR}\rvert\lvert e^{\pi t}\rvert}{\lvert e^{R}\rvert\lvert e^{i\pi t}\rvert-\lvert e^{-R}\rvert\lvert e^{-i\pi t}\rvert}\Big\rvert \]
  \[  \leq \lvert (e^{i-1})^R\rvert\Big\lvert\frac{\lvert e^{-\pi t}\rvert+\lvert e^{-2iR}\rvert\lvert e^{\pi t}\rvert}{\lvert e^{i\pi t}\rvert-\lvert e^{-2R}\rvert\lvert e^{-i\pi t}\rvert}\Big\rvert \]  
    \[  \leq  (e^{-R})\Big\lvert\frac{\lvert e^{-\pi t}\rvert+\lvert e^{-2iR}\rvert\lvert e^{\pi t}\rvert}{\lvert e^{i\pi t}\rvert-\lvert e^{-2R}\rvert\lvert e^{-i\pi t}\rvert}\Big\rvert \] 
  As R $\to \infty $, this value goes to zero.
  
  A similar case can be made for $C_6$.
  
 Taking the curve $C_4$,
 \[z = t+ 2\pi i \]
 where t goes from $\epsilon$ to R.
\[\int_{\epsilon + \pi i}^{R+ \pi i} \frac{\sin z}{\sinh z}dz =\int_{\epsilon + \pi i}^{R+\pi i} \frac{\sin (t+ \pi i )}{\sinh (t+ \pi i )}dz  \]
\[=\int_{\epsilon + \pi i}^{R+\pi i} \frac{\sin (t)cos(\pi i)+sin(\pi i)cos(t)}{\sinh (t)cosh(\pi i)+sinh(\pi i)cosh(t)}dz\]
\[=-\int_{\epsilon}^{R} \frac{\sin (z)cos(\pi i)}{\sinh (z)}dz\]
Similarly, on $C_6$, we can show,
\[\int_{-\epsilon + \pi i}^{-R+ \pi i} \frac{\sin z}{\sinh z}dz=-cos(\pi i)\int_{-\epsilon}^{-R} \frac{\sin (z)}{\sinh (z)}dz\]

So, we have now,
 \[\left(\int_{-R}^{-\epsilon} +\int_{C_0}+\int_{\epsilon}^{R} +cos(\pi i) \int_{\epsilon}^{R} +\int_{C_{\pi}} +cos(\pi i) \int_{-R}^{-\epsilon}\right) f(z)dz= 0\]
  \[(1+cos(\pi i))\left(\int_{-R}^{-\epsilon} +\int_{\epsilon}^{R} \right) f(z)dz=-\left(\int_{C_0} +\int_{C_{\pi}}\right) f(z)dz\]
 As R goes to $\infty$,
 \[(1+cos(\pi i)) I_p = -\left(\int_{C_0} +\int_{C_{\pi}}\right) f(z)dz \]
 At 0, the function has a removable singularity. So to find the integral over $C_0$ using Theorem 4.3.1(b)(Baby Limit theorem), we need to change it into a form which has simple poles.
 
 \[\int_{C_0} f(z)dz= -\int_{C_0} \frac{e^{iz}-e^{-iz}}{2i sinh(z)} dz = -\int_{C_0} \frac{e^{iz}}{2i sinh(z)} dz+ \int_{C_0} \frac{e^{-iz}}{2i sinh(z)} dz\]
  The minus sign comes from the integral being in the clockwise direction.
 Both the terms have a simple pole at z =0 since
 \[ \frac{1}{sinh(z)}= \frac{1}{z} - \frac{z}{6}+ \ldots\]
  \[\int_{C_0} f(z)dz= \pi i \frac{1}{2i} - \pi i \frac{1}{2i} =0\]
  At $\pi i$, the function has a simple pole and as 
   \[ \frac{1}{sinh(z-\pi i)}= -\frac{1}{z-\pi i} + \frac{z-\pi i}{6}+ \ldots\]
  We can use Theorem 4.3.1(b) directly to obtain
    \[\int_{C_{\pi}} f(z)dz=  - (-\pi i sin(\pi i)) =\pi i sin(\pi i) \]
   Substituting that in 
    \[(1+cos(\pi i)) I_p = -\left(\int_{C_0} +\int_{C_{\pi}}\right) f(z)dz \],
     \[I_p = - \pi i \frac{sin(\pi i)}{(1+cos(\pi i)) } =-\pi i tan(\frac{\pi i}{2})=\pi  tanh(\frac{\pi }{2})  \]

	\end{enumerate} 
\end{document}